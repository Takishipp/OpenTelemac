\chapter{Organisation of the \telemacsystem activity}

\section{Structure of the \telemacsystem activity}

\subsection{General Organisation}

The \telemacsystem is managed by a Consortium composed of two main instances:
\begin{itemize}
\item The Steering Committee,
\item The Technical and Scientific Committee.
\end{itemize}

The main functions of the Steering Committee are as follows:
\begin{itemize}
\item To seek advice from the Technical and Scientific Committee and so to
decide on the main strategic technical, technological and scientific focuses of
the Consortium for the medium-to-long term,
\item To seek advice from the Technical and Scientific Committee and so to
decide, choose and prioritise action items to be included in the Development
Plan, whether the action items are proposed by the Members of the Consortium or
by third Parties,
\item To define modes of implementation of the Development Plan and modes of
integration in the SOFTWARE of the deliverables of the Development Plan calling
on resources either within the Members of the Consortium or on the Experts in
Confidence,
\item To make sure that the deliverables of the Development Plan are of desired
quality, including but not limited to compatible source code, documentation,
cases for benchmarking and technical approval,
\item To identify, define and validate the strategy for extracting added value
from the Prior Knowledge and promote the inclusion of Prior Knowledge in the
SOFTWARE whether it resides within the Consortium or not,
\item To ensure that the present Membership Agreement is performed properly
and, if necessary, propose endorsements aimed at improving operation of the
Membership Agreement,
\item To grant/revoke voting rights to Member Representatives,
\item To include/exclude a Member to/from the Consortium subject to the
Duration and termination of a Member's membership.
\end{itemize}

The main functions of the Chairperson of the Steering Committee are as follows:
\begin{itemize}
\item To gather action items, feature wish lists and developments provided by
Member Representatives and third Parties and to seek development ideas from
other open source software communities,
\item To convene and preside over the meetings of the Steering Committee,
\item To formalise and report minutes of the meetings of the Steering
Committee,
\item To formalise and report the Development Plan.
\end{itemize}

The main functions of the Technical and Scientific Committee are as follows:
\begin{itemize}
\item To advise the Steering Committee in its scientific and technical focuses,
\item To diagnose for the Steering Committee the admissibility of the
developments supplied to the Steering Committee by the Members of the
Consortium or by third Parties,
\item To enlighten the Steering Committee on the prioritisation to be done and
the benchmarking and technical approval plans to be implemented.
\end{itemize}

The main functions of the Chairperson of the Technical and Scientific Committee are as follows:
\begin{itemize}
\item To convene and preside over the meetings of the Representative Members of
the Technical and Scientific Committee,
\item To formalise and report minutes of the meetings of the Representative
Members of the Technical and Scientific Committee and gather appropriate
references, technical notes or scientific evidences supporting its conclusions,
\item To synthesise answers to requests submitted via the Chairperson of the
Steering Committee.
\end{itemize}

\section{Step in \telemacsystem activities}

The \telemacsystem activity is divided in three main categories directing the life of a
version of the software, from its creation to its exploitation and finally to
its archiving (See Development Plan in Appendix \ref{devplan} for definition of those
notions):
\begin{itemize}
\item \textbf{Development activity}: concerns any modification on the code, its
documentation and the associated tools, either it is an evolution, a
correction, a reorganisation.
\item \textbf{Verification and validation activity}: every development must be
verified by the verification and validation tests. A major version or
production version (See Appendix \ref{devplan}) must run all the cases.
\item \textbf{Exploitation activity}: concern the distribution, formation,
support, maintenance of the different version and the removal of exploitation
as well as the archiving.
\end{itemize}

\subsection{Responsibilities}

The affectation for the different responsibilities are given in the document
"Nominative list of the people of \telemacsystem" in Appendix \ref{peopleoftelma}.

\section{Development activity}

A development represent any intervention on the source code of \telemacsystem, its
documentation or its tools (i.e. any element under the configuration, see the
development plan in Appendix \ref{devplan}). The development on the code can be
subdivided into different categories: 
\begin{itemize}
\item \textbf{Evolution maintenance}: The evolution maintenance changes the
software to break some of the limitation he had (performance, strength,
precision, adaptation to demands \ldots) or to add new functionalities to the
software to answer new demands or requests.
\item \textbf{Corrective maintenance}: The corrective maintenance removes a bug
in the code or the documentation.
\item \textbf{Adaptive maintenance}: The adaptive maintenance adapts the
software when its environment changes (Operating system, compilers, new type of
clusters\ldots) to insure the software is still running on that environment.
\end{itemize}

Those three categories imply an integration process in the development version
of \telemacsystem.

The main actors are:
\begin{itemize}
\item The development team from each member of the consortium.
\item The developer from outside firms.
\item The Open-source community, academic and industrial partners.
\end{itemize}
The process attached to those activities are described in the development plan
in Appendix \ref{devplan}

\subsection{Processes, tools and rules}

The processes to follow for each development activities are described in the
development plan in Appendix \ref{devplan}. Following those processes ensure
traceability, non-regression, portability, verification, validation and
re-usability of the functions developed and integrated in \telemacsystem.

Those rules are based on the following principles:
\begin{itemize}
\item Keep the programming coherent with the existing one.
\item Always base yourself on the latest version of the code (the development
version) when creating new files.
\item Communicate with the developer community on the evolution of the code structure.
\end{itemize}

Every new development must have a ticket on the \telemacsystem project manager CUE
(\url{http://cue.opentelemac.org}) may it be a evolution, corrective or
adaptive maintenance in order to organize and prioritize development made in
\telemacsystem.

All those development must be validated by the associated CH and the PM.

\subsection{Evolution maintenance}

The developments realised for any maintenance are under the responsibility of
the developer who was given this assignment by the CHs.  They are handled by
following the processes described in the development plan in Appendix
\ref{devplan}.

The final integration of the development of a maintenance is in
charge of the associated CH and must follow the process described in the
development plan in Appendix \ref{devplan}

\section{Verification and validation activity}
\label{vv}
The verification and validation activity aims to verify the implementation on
an algorithmic level or on the complete model using verification and validation
test cases (even if a small part of the code was impacted as it can be for the
corrective and adaptive maintenance). It is complementary (even a bit
redundant) with the test realised during the implementation (See the
development plan in Appendix \ref{devplan}). We distinguish them as follow:
\begin{itemize}
\item \textbf{Verification}: The verification aims to establish that the code
solves efficiently the mathematical problem he is supposed to resolve (and that
is described in the specification of the concern algorithm). The verification
focuses on the quality of the implementation from a computer science and
theoretical point of view. The answer to the verification process is a binary
one: the code does or does not respond to the demands. To do that the
verification is using an analytical solution of the model or a solution based
on measurements with the possible addition of source terms to the equation. The
process is then most of the time to compare the numerical solution with a
reference solution on meshes with increasing mesh refinement. This verifies
that the algorithm converges in mesh and gives the level of precision
associated.
\item \textbf{Validation}: The validation aims to see to what level the code is
able to answer to a physical problem in an given application domain.  The
validation focuses on the ability of the code to predict event from a physical
point of view. The answer to this process is not binary. The evaluation is made
through a comparison between a numerical solution and an experimental solution.
We distinguish the validation with separated effects, which focuses on an
isolated physical phenomenon and the full validation which focuses on an
industrial problem (which can be simplify to get access to experimental
measurements) composed of multiple physical phenomenon.
\item \textbf{Qualification}: The qualification aims to establish the perimeter
of action of the code by using heavily validated complexed cases. This action
is not done by the developers, it is in charge of main users. The creation of a
methodology note is part of the qualification process. The qualification can
keep track of the performance of the code as well as the non-regression of the
test cases.
\end{itemize}

\section{Release and distribution activity}

\subsection{Release, reception, removal}

Between the different versions of \telemacsystem, we distinguish the release of a major
version or production version from the other versions (minor, development
release).

The different types of version are given in the development plan in Appendix
\ref{devplan}

The release of a new major version of \telemacsystem is decided by the chief of
department hosting the \telemacsystem project. The list of what is concerned by this
new version is written by the CHs in the version note. This note describes:
\begin{itemize}
\item The code elements that compose the software,
\item The documentation,
\item The code environment,
\item The major modifications since version $n-1$.
\end{itemize}

This version note formalises the quality insurance of the major version.

For the minor version release of \telemacsystem the creation of this version note is
not necessary instead the information about the modifications in the code are
written in the release note. Even though the minor version still has to follow
the quality processes followed by a major version.

The development version of \telemacsystem is not formally released. Never the less it
is compiled and a few test cases are run nightly. The last development version
is available on the \telemacsystem svn
(\url{http://svn.opentelemac.org/svn/opentelemac/trunk}).

The process for the removal of older stable version is described in the
development plan in Appendix \ref{devplan}.

\subsection{Distribution}

\textbf{Distribution in EDF R\&D}

The major and minor versions of \telemacsystem are installed on the NOE server on an
area dedicated to the project. The DSPIT handles the backup of those data.

The test cases are available (at least in reading) on the NOE server. The
validation and verification is made on an external server based in
H.R.Wallingford (UK). The environments on which the code was verified and
validated are listed in the version note.

\textbf{Distribution in EDF}

The distribution outside of the R\&D is under the responsibility of the \telemacsystem
Chain Handler. The environment guide written by the Chain Handl. which
contains an help for the installation of the code.

\textbf{Open source distribution}

All the stable version of \telemacsystem, the development version and the tools are
under open source license (GPL for \telemacsystem modules except for the \bief which is
under LGPL) and are available on the website \url{http://www.opentelemac.org}
which is the main access for all information about \telemacsystem.

The open source version are tested with on Windows 7 and on the following Linux
Dristribution: Opensuse, Fedora, Ubuntu, Debian. It is also tested using the
NAG Compiler, the intel Compiler and the gnu compiler.

\textbf{Protection}

The software is protected by the label \tel since 05/02/96 with the number
INPI 96/623 748.

\subsection{Relation with the \telemacsystem users}

The services handled by the \telemacsystem project and available to users are:
\begin{itemize}
\item A technical assistance: Forum, bug tracker, guides, documentation.
\item Meeting with users: internally in EDF or externally (User Conference)
\end{itemize}

All those services are available on the \telemacsystem website
(\url{http://www.opentelemac.org}). 
The developers responding to the forum are not bound to respond to the forum,
they do that on their available time.

\textbf{Technical assistance}

The organisation of the technical assistance is divided into two levels:
\begin{itemize}
\item \textbf{First level assistance}\\
This assistance can concern either the installation or the usage of the
software. This assistance can lead to a demand for an evolution of the code or
the discovery of a bug. A ticket must then be created on the development
project manager CUE (\url{http://cue.opentelemac.org}). The formalisation of
those demands allows:
\begin{itemize}
\item A better repartition of resources by the project.
\item The follow up by the user of the demand and allows him to interact with
the developer handling the demand.
\end{itemize}
\item \textbf{Second level assistance}\\
This assistance is made by the development team of the project \telemacsystem. It is an
anticipated action aimed to create documentation and didactic software (user
and developer documentation, online documentation, tutorials, FAQs...).
\end{itemize}

The assistance activity can lead to a ticket on the CUE that must follow the
development rules established in the development plan in Appendix
\ref{devplan}, if the problem is deemed worth it.

The actors of this activity are:
\begin{itemize}
\item The users: They are telling us their needs, helping in the specification and
the validation of the code. They can also respond to some questions on the
forum.
\item The people allowed to submit development demands more specifically the
development team (See the development plan in Appendix \ref{devplan}): They answer to
the forum, the CUE. They subcontract some of the work then they handle the
dialog with the subcontract.
\item The Kernel Handler: handles the CUE, its maintenance and its good use.
\end{itemize}

\textbf{Meeting with users}

The meeting with user orbits around the following element:
\begin{itemize}
\item \textbf{Organisation of the \telemacsystem User Conference}: The organisation of
the conference is explained in Appendix \ref{tuc1} and in the document
describing the organisation of the \telemacsystem activity in Appendix
\ref{fullorga}.
\end{itemize}

\subsection{Relation with partners}

The partners (like for example the members of the \telemacsystem consortium) must
follow the same rules for the development as described in the development plan
in Appendix \ref{devplan}.

\section{Control of sub-contract}

Any developer from the development team can sub-contract some of his
development like:
\begin{itemize}
\item The corrective, adaptive, evolution maintenance.
\item Action on the documentation.
\item Action of verification and validation.
\end{itemize}

In this case the process and the rules to follow are the same as an internal
development described in the development plan in Appendix \ref{devplan}. The
developer ordering this sub-contract is in charge of the follow-up of the work
and its future maintenance.

Other actions can be sub-contracted:
\begin{itemize}
\item Action on the website.
\item Action of support.
\end{itemize}
