%--------------------------------------------------------------------------------
\chapter{Description}
%--------------------------------------------------------------------------------
%
The aim of this module is to produce generic functions to read/write meshes,
data and boundary conditions in \telemacsystem regardless of the file format. For that
purpose a new module, called Hermes, was created.
%
%--------------------------------------------------------------------------------
\chapter{User Manual}
%--------------------------------------------------------------------------------
%
Three formats are now available in \telemacsystem:
\begin{itemize}
\item SERAFIN, \telemacsystem own format, a binary containing the mesh and the results
information, all the real are in single precision.
\item SERAFIND, same as above but the real are in double precision.
\item MED, a binary format based on hdf5, format used by the Salome platform.
This format requires to install additional libraries and add specific option
to the \telemacsystem installation therefore it not installed by default.
\end{itemize}

In order to set the file format, a keyword is defined for every file, for
example the keyword for "geometry file" is "geometry file format".

All the files used for a simulation must be in the same format, this is due to
the fact that the same boundary file is used for all the files and the boundary
file is format-dependant. Only SERAFIN an SERAFIND can coexist as they are
using the same format for their boundary file.
%--------------------------------------------------------------------------------
\section{List of functions}
%--------------------------------------------------------------------------------
\label{listfunc}
\subsection{Mesh functions}
\begin{lstlisting}
subroutine open_mesh(fformat,file_name,file_id,openmode,ierr)
!
!brief    opens a mesh file
!
!~~~~~~~~~~~~~~~~~~~~~~~~~~~~~~~~~~~~~~~~~~~~~~~~~~~~~~~~~~~~~~~~~~~~~~~
!| fformat        |-->| format of the file
!| file_name      |-->| name of the file
!| file_id        |-->| file descriptor
!| openmode       |-->| one of the following value 'read','write','readwrite'
!| ierr           |<--| 0 if no error during the opening
!~~~~~~~~~~~~~~~~~~~~~~~~~~~~~~~~~~~~~~~~~~~~~~~~~~~~~~~~~~~~~~~~~~~~~~~
  character(len=8),  intent(in)    :: fformat
  character(len=*), intent(in)   :: file_name
  integer,            intent(in)   :: file_id
  character(len=9), intent(in)     :: openmode
  integer, intent(out)             :: ierr
!
end subroutine
\end{lstlisting}
%
\begin{lstlisting}
subroutine close_mesh (fformat,file_id,ierr)
!
!brief    closes a mesh file
!
!~~~~~~~~~~~~~~~~~~~~~~~~~~~~~~~~~~~~~~~~~~~~~~~~~~~~~~~~~~~~~~~~~~~~~~~
!| fformat        |-->| format of the file
!| file_id        |-->| file descriptor
!| ierr           |<--| 0 if no error during the opening
!~~~~~~~~~~~~~~~~~~~~~~~~~~~~~~~~~~~~~~~~~~~~~~~~~~~~~~~~~~~~~~~~~~~~~~~
  character(len=8),  intent(in)    :: fformat
  integer,            intent(in)   :: file_id
  integer, intent(out)             :: ierr
!
end subroutine
\end{lstlisting}
%
\begin{lstlisting}
subroutine get_mesh_title (fformat,fid,title,ierr)
!
!brief    returns the title from a mesh file
!
!~~~~~~~~~~~~~~~~~~~~~~~~~~~~~~~~~~~~~~~~~~~~~~~~~~~~~~~~~~~~~~~~~~~~~~~
!| fformat        |-->| format of the file
!| fid            |-->| file descriptor
!| title          |<->| title of the mesh file
!| ierr           |<--| 0 if no error during the opening
!~~~~~~~~~~~~~~~~~~~~~~~~~~~~~~~~~~~~~~~~~~~~~~~~~~~~~~~~~~~~~~~~~~~~~~~
!
  character(len=8),  intent(in)  :: fformat
  integer,           intent(in)  :: fid
  character(len=80), intent(inout) :: title
  integer,           intent(out) :: ierr
!
end subroutine
\end{lstlisting}
%
\begin{lstlisting}
subroutine get_mesh_date (fformat,fid,date,ierr)
!
!brief    returns the date of the mesh file
!
!~~~~~~~~~~~~~~~~~~~~~~~~~~~~~~~~~~~~~~~~~~~~~~~~~~~~~~~~~~~~~~~~~~~~~~~
!| fformat        |-->| format of the file
!| fid            |-->| file descriptor
!| date           |<->| the date
!| ierr           |<--| 0 if no error during the opening
!~~~~~~~~~~~~~~~~~~~~~~~~~~~~~~~~~~~~~~~~~~~~~~~~~~~~~~~~~~~~~~~~~~~~~~~
!
  character(len=8), intent(in)  :: fformat
  integer,          intent(in)  :: fid
  integer,          intent(inout) :: date(6)
  integer,          intent(out) :: ierr
!
end subroutine
\end{lstlisting}
%
\begin{lstlisting}
subroutine get_mesh_nelem (fformat,fid,typ_elem,nelem,ierr)
!
!brief    returns the number of elements of type typ_elem in the mesh file
!
!~~~~~~~~~~~~~~~~~~~~~~~~~~~~~~~~~~~~~~~~~~~~~~~~~~~~~~~~~~~~~~~~~~~~~~~
!| fformat        |-->| format of the file
!| fid            |-->| file descriptor
!| typ_elem       |-->| type of the element
!| nelem          |<->| the number of elements
!| ierr           |<--| 0 if no error during the opening
!~~~~~~~~~~~~~~~~~~~~~~~~~~~~~~~~~~~~~~~~~~~~~~~~~~~~~~~~~~~~~~~~~~~~~~~
!
  character(len=8), intent(in)  :: fformat
  integer,          intent(in)  :: fid
  integer,          intent(in)  :: typ_elem
  integer,          intent(inout) :: nelem
  integer,          intent(out) :: ierr
!
end subroutine
\end{lstlisting}
%
\begin{lstlisting}
subroutine get_mesh_ndp (fformat,fid,typ_elem,ndp,ierr)
!
!brief    returns the number of point per element of type typ_elem
!
!~~~~~~~~~~~~~~~~~~~~~~~~~~~~~~~~~~~~~~~~~~~~~~~~~~~~~~~~~~~~~~~~~~~~~~~
!| fformat        |-->| format of the file
!| fid            |-->| file descriptor
!| typ_elem       |-->| type of the element
!| ndp            |<->| the number of point per element
!| ierr           |<--| 0 if no error during the opening
!~~~~~~~~~~~~~~~~~~~~~~~~~~~~~~~~~~~~~~~~~~~~~~~~~~~~~~~~~~~~~~~~~~~~~~~
!
  character(len=8), intent(in)  :: fformat
  integer,          intent(in)  :: fid
  integer,          intent(in)  :: typ_elem
  integer,          intent(inout) :: ndp
  integer,          intent(out) :: ierr
!
end subroutine
\end{lstlisting}
%
\begin{lstlisting}
subroutine get_mesh_ikle (fformat,fid,typ_elem,ikle,nelem,ndp,ierr)
!
!brief    returns the connectivity table for 
!+        the element of type typ_elem in the mesh
!+        will do nothing if there are no element of typ_elem in the mesh
!
!~~~~~~~~~~~~~~~~~~~~~~~~~~~~~~~~~~~~~~~~~~~~~~~~~~~~~~~~~~~~~~~~~~~~~~~
!| fformat        |-->| format of the file
!| fid            |-->| file descriptor
!| typ_elem       |-->| type of the element
!| ikle           |<->| the connectivity table
!| nelem          |-->| number of elements
!| ndp            |-->| number of points per element
!| ierr           |<--| 0 if no error during the opening
!~~~~~~~~~~~~~~~~~~~~~~~~~~~~~~~~~~~~~~~~~~~~~~~~~~~~~~~~~~~~~~~~~~~~~~~
!
  character(len=8), intent(in)  :: fformat
  integer,          intent(in)  :: fid
  integer,          intent(in)  :: typ_elem
  integer,          intent(in)  :: nelem
  integer,          intent(in)  :: ndp
  integer,          intent(inout) :: ikle(nelem*ndp)
  integer,          intent(out) :: ierr
!
end subroutine
\end{lstlisting}
%
\begin{lstlisting}
subroutine get_mesh_npoin (fformat,fid,typ_elem,npoin,ierr)
!
!brief    returns the number of point for the given element type in the mesh file
!
!~~~~~~~~~~~~~~~~~~~~~~~~~~~~~~~~~~~~~~~~~~~~~~~~~~~~~~~~~~~~~~~~~~~~~~~
!| fformat        |-->| format of the file
!| fid            |-->| file descriptor
!| typ_elem       |-->| type of the element
!| npoin          |<->| the number of points
!| ierr           |<--| 0 if no error during the opening
!~~~~~~~~~~~~~~~~~~~~~~~~~~~~~~~~~~~~~~~~~~~~~~~~~~~~~~~~~~~~~~~~~~~~~~~
!
  character(len=8), intent(in)  :: fformat
  integer,          intent(in)  :: fid
  integer,          intent(in)  :: typ_elem
  integer,          intent(inout) :: npoin
  integer,          intent(out) :: ierr
!
end subroutine
\end{lstlisting}
%
\begin{lstlisting}
subroutine get_mesh_nplan (fformat,fid,nplan,ierr)
!
!brief    returns the number of layers
!
!~~~~~~~~~~~~~~~~~~~~~~~~~~~~~~~~~~~~~~~~~~~~~~~~~~~~~~~~~~~~~~~~~~~~~~~
!| fformat        |-->| format of the file
!| fid            |-->| file descriptor
!| nplan          |<->| the number of layers
!| ierr           |<--| 0 if no error during the opening
!~~~~~~~~~~~~~~~~~~~~~~~~~~~~~~~~~~~~~~~~~~~~~~~~~~~~~~~~~~~~~~~~~~~~~~~
!
  character(len=8), intent(in)  :: fformat
  integer,          intent(in)  :: fid
  integer,          intent(inout) :: nplan
  integer,          intent(out) :: ierr
!
end subroutine
\end{lstlisting}
%
\begin{lstlisting}
subroutine get_mesh_dimension (fformat,fid,ndim,ierr)
!
!brief    returns the number of dimensions of the space
!
!~~~~~~~~~~~~~~~~~~~~~~~~~~~~~~~~~~~~~~~~~~~~~~~~~~~~~~~~~~~~~~~~~~~~~~~
!| fformat        |-->| format of the file
!| fid            |-->| file descriptor
!| ndim           |<->| number of dimension
!| ierr           |<--| 0 if no error during the opening
!~~~~~~~~~~~~~~~~~~~~~~~~~~~~~~~~~~~~~~~~~~~~~~~~~~~~~~~~~~~~~~~~~~~~~~~
!
  character(len=8), intent(in)  :: fformat
  integer,          intent(in)  :: fid
  integer,          intent(inout) :: ndim
  integer,          intent(out) :: ierr
!
end subroutine
\end{lstlisting}
%
\begin{lstlisting}
subroutine get_mesh_coord (fformat,fid,jdim,ndim,npoin,coord,ierr)
!
!brief    returns the coordinates for the given dimension
!
!~~~~~~~~~~~~~~~~~~~~~~~~~~~~~~~~~~~~~~~~~~~~~~~~~~~~~~~~~~~~~~~~~~~~~~~
!| fformat        |-->| format of the file
!| fid            |-->| file descriptor
!| jdim           |-->| dimension number
!| ndim           |-->| number of dimension of the mesh
!| npoin          |-->| total number of nodes
!| coord          |<->| local to global numbering array
!| ierr           |<--| 0 if no error during the opening
!~~~~~~~~~~~~~~~~~~~~~~~~~~~~~~~~~~~~~~~~~~~~~~~~~~~~~~~~~~~~~~~~~~~~~~~
!
  character(len=8), intent(in)  :: fformat
  integer,          intent(in)  :: fid, jdim, ndim, npoin
  double precision, intent(inout) :: coord(npoin)
  integer,          intent(out) :: ierr
!
end subroutine
\end{lstlisting}
%
\begin{lstlisting}
subroutine get_mesh_knolg (fformat,fid,knolg,npoin,ierr)
!
!brief    returns the local to global numbering array
!
!~~~~~~~~~~~~~~~~~~~~~~~~~~~~~~~~~~~~~~~~~~~~~~~~~~~~~~~~~~~~~~~~~~~~~~~
!| fformat        |-->| format of the file
!| fid            |-->| file descriptor
!| knolg          |<->| local to global numbering array
!| npoin          |-->| number of nodes
!| ierr           |<--| 0 if no error during the opening
!~~~~~~~~~~~~~~~~~~~~~~~~~~~~~~~~~~~~~~~~~~~~~~~~~~~~~~~~~~~~~~~~~~~~~~~
!
  character(len=8), intent(in)  :: fformat
  integer,          intent(in)  :: fid
  integer,          intent(in)  :: npoin
  integer,          intent(inout) :: knolg(npoin)
  integer,          intent(out) :: ierr
!
end subroutine
\end{lstlisting}
%
\begin{lstlisting}
subroutine get_mesh_nptir (fformat,fid,nptir,ierr)
!
!brief    returns the number of interface point
!
!~~~~~~~~~~~~~~~~~~~~~~~~~~~~~~~~~~~~~~~~~~~~~~~~~~~~~~~~~~~~~~~~~~~~~~~
!| fformat        |-->| format of the file
!| fid            |-->| file descriptor
!| nptir          |<->| number of interface point
!| ierr           |<--| 0 if no error during the opening
!~~~~~~~~~~~~~~~~~~~~~~~~~~~~~~~~~~~~~~~~~~~~~~~~~~~~~~~~~~~~~~~~~~~~~~~
!
  character(len=8), intent(in)  :: fformat
  integer,          intent(in)  :: fid
  integer,          intent(inout) :: nptir
  integer,          intent(out) :: ierr
!
end subroutine
\end{lstlisting}
%
\subsection{Boundary functions}
%
\begin{lstlisting}
subroutine open_bnd(fformat,file_name,file_id,openmode,ierr)
!
!brief    opens a boundary file
!
!~~~~~~~~~~~~~~~~~~~~~~~~~~~~~~~~~~~~~~~~~~~~~~~~~~~~~~~~~~~~~~~~~~~~~~~
!| fformat        |-->| format of the file
!| file_name      |-->| name of the file
!| file_id        |-->| file descriptor of the "mesh" file
!| openmode       |-->| one of the following value 'read','write','readwrite'
!| ierr           |<--| 0 if no error during the opening
!~~~~~~~~~~~~~~~~~~~~~~~~~~~~~~~~~~~~~~~~~~~~~~~~~~~~~~~~~~~~~~~~~~~~~~~
  character(len=8),  intent(in)    :: fformat
  character(len=*), intent(in)   :: file_name
  integer,            intent(in)   :: file_id
  character(len=9), intent(in)     :: openmode
  integer, intent(out)             :: ierr
!
end subroutine
\end{lstlisting}
%
\begin{lstlisting}
subroutine close_bnd (fformat,file_id,ierr)
!
!brief    closes a boundary file
!
!~~~~~~~~~~~~~~~~~~~~~~~~~~~~~~~~~~~~~~~~~~~~~~~~~~~~~~~~~~~~~~~~~~~~~~~
!| fformat        |-->| format of the file
!| file_id        |-->| file descriptor of the "mesh" file
!| ierr           |<--| 0 if no error during the opening
!~~~~~~~~~~~~~~~~~~~~~~~~~~~~~~~~~~~~~~~~~~~~~~~~~~~~~~~~~~~~~~~~~~~~~~~
  character(len=8),  intent(in)    :: fformat
  integer,            intent(in)   :: file_id
  integer, intent(out)             :: ierr
!
end subroutine
\end{lstlisting}
%
\begin{lstlisting}
subroutine get_bnd_ipobo (fformat,fid,npoin,nelebd,typ_bnd_elem,ipobo,ierr)
!
!brief    returns an array containing 
!+        1 if a point is a boundary point 0 otherwise
!
!~~~~~~~~~~~~~~~~~~~~~~~~~~~~~~~~~~~~~~~~~~~~~~~~~~~~~~~~~~~~~~~~~~~~~~~
!| fformat        |-->| format of the file
!| fid            |-->| file descriptor
!| npoin          |-->| total number of nodes
!| nelebd         |-->| total number of boundary elements
!| typ_bnd_elem   |-->| type of the boundary element
!| ipobo          |<->| an array containing
!|                |   | 1 if a point is a boundary point 0 otherwise
!| ierr           |<--| 0 if no error during the opening
!~~~~~~~~~~~~~~~~~~~~~~~~~~~~~~~~~~~~~~~~~~~~~~~~~~~~~~~~~~~~~~~~~~~~~~~
!
  character(len=8), intent(in)  :: fformat
  integer,          intent(in)  :: fid, npoin, nelebd, typ_bnd_elem
  integer,          intent(inout) :: ipobo(npoin)
  integer,          intent(out) :: ierr
!
end subroutine
\end{lstlisting}
%
\begin{lstlisting}
subroutine get_bnd_nbor (fformat,fid,typ_bnd_elem,nptfr,nbor,ierr)
!
!brief    returns an array containing 
!+        the association of boundary numbering to mesh numbering
!
!~~~~~~~~~~~~~~~~~~~~~~~~~~~~~~~~~~~~~~~~~~~~~~~~~~~~~~~~~~~~~~~~~~~~~~~
!| fformat        |-->| format of the file
!| fid            |-->| file descriptor
!| typ_bnd_elem   |-->| type of the boundary element
!| nptfr          |-->| number of boundary points
!| nbor           |<->| an array containing the numbering in the mesh 
!|                |   | of all boundary points
!| ierr           |<--| 0 if no error during the opening
!~~~~~~~~~~~~~~~~~~~~~~~~~~~~~~~~~~~~~~~~~~~~~~~~~~~~~~~~~~~~~~~~~~~~~~~
!
  character(len=8), intent(in)  :: fformat
  integer,          intent(in)  :: fid, nptfr, typ_bnd_elem
  integer,          intent(inout) :: nbor(nptfr)
  integer,          intent(out) :: ierr
!
end subroutine
\end{lstlisting}
%
\begin{lstlisting}
subroutine get_bnd_ikle (fformat,fid,typ_bnd_elem,nelebd,ndp,ikle_bnd,ierr)
!
!brief    reads the connectivity of the boundary elements
!
!~~~~~~~~~~~~~~~~~~~~~~~~~~~~~~~~~~~~~~~~~~~~~~~~~~~~~~~~~~~~~~~~~~~~~~~
!| fformat        |-->| format of the file
!| fid            |-->| file descriptor
!| typ_bnd_elem   |-->| type of the boundary elements
!| nelebd         |-->| number of boundary elements
!| ndp            |-->| number of points per element
!| ikle_bnd       |<->| the connectivity of the boundary elements
!| ierr           |<--| 0 if no error during the opening
!~~~~~~~~~~~~~~~~~~~~~~~~~~~~~~~~~~~~~~~~~~~~~~~~~~~~~~~~~~~~~~~~~~~~~~~
!
  character(len=8), intent(in) :: fformat
  integer, intent(in) :: fid,typ_bnd_elem,nelebd,ndp
  integer, intent(inout) :: ikle_bnd(ndp*nelebd)
  integer, intent(out) :: ierr
!
end subroutine
\end{lstlisting}
%
\begin{lstlisting}
subroutine get_bnd_npoin (fformat,fid,type_bnd_elem,nptfr,ierr)
!
!brief    returns the number of boundary points
!
!~~~~~~~~~~~~~~~~~~~~~~~~~~~~~~~~~~~~~~~~~~~~~~~~~~~~~~~~~~~~~~~~~~~~~~~
!| fformat        |-->| format of the file
!| fid            |-->| file descriptor
!| type_bnd_elem  |-->| type of the boundary elements
!| nptfr          |<->| number of boundary points
!| ierr           |<--| 0 if no error during the opening
!~~~~~~~~~~~~~~~~~~~~~~~~~~~~~~~~~~~~~~~~~~~~~~~~~~~~~~~~~~~~~~~~~~~~~~~
!
  character(len=8), intent(in)  :: fformat
  integer,          intent(in)  :: fid
  integer,          intent(in)  :: type_bnd_elem
  integer,          intent(inout) :: nptfr
  integer,          intent(out) :: ierr
!
end subroutine
\end{lstlisting}
%
\begin{lstlisting}
subroutine get_bnd_nelem (fformat,fid,type_bnd_elem,nelem,ierr)
!
!brief    reads the number of boundary elements
!
!~~~~~~~~~~~~~~~~~~~~~~~~~~~~~~~~~~~~~~~~~~~~~~~~~~~~~~~~~~~~~~~~~~~~~~~
!| fformat        |-->| format of the file
!| fid            |-->| file descriptor
!| type_bnd_elem  |-->| type of the boundary elements
!| nelem          |<->| number of boundary elements
!| ierr           |<--| 0 if no error during the opening
!~~~~~~~~~~~~~~~~~~~~~~~~~~~~~~~~~~~~~~~~~~~~~~~~~~~~~~~~~~~~~~~~~~~~~~~
!
  character(len=8), intent(in)  :: fformat
  integer,          intent(in)  :: fid
  integer,          intent(in)  :: type_bnd_elem
  integer,          intent(inout) :: nelem
  integer,          intent(out) :: ierr
!
end subroutine
\end{lstlisting}
%
\begin{lstlisting}
subroutine get_bnd_value (fformat,fid,typ_bnd_elem,nelebd,value,nptfr,nbor,ierr)
!
!brief    returns an array containing the boundary type for each boundary point
!
!~~~~~~~~~~~~~~~~~~~~~~~~~~~~~~~~~~~~~~~~~~~~~~~~~~~~~~~~~~~~~~~~~~~~~~~
!| fformat        |-->| format of the file
!| fid            |-->| file descriptor
!| typ_bnd_elem   |-->| type of the boundary elements
!| nelebd         |-->| number of boundary elements
!| value          |<->| type of boundary for each point
!| nptfr          |-->| number of boundary points
!| nbor           |-->| boundary to global numbering array
!| ierr           |<--| 0 if no error during the opening
!~~~~~~~~~~~~~~~~~~~~~~~~~~~~~~~~~~~~~~~~~~~~~~~~~~~~~~~~~~~~~~~~~~~~~~~
!
  character(len=8), intent(in)  :: fformat
  integer,          intent(in)  :: fid
  integer,          intent(in)  :: typ_bnd_elem
  integer,          intent(in)  :: nelebd
  integer,          intent(in)  :: nptfr
  integer,          intent(inout) :: value(nptfr)
  integer,          intent(in) :: nbor(nptfr)
  integer,          intent(out) :: ierr
!
end subroutine
\end{lstlisting}
%
\subsection{Data functions}
%
\begin{lstlisting}
subroutine get_data_nvar (fformat,fid,nvar,ierr)
!
!brief    returns the number of variables in the mesh file
!
!~~~~~~~~~~~~~~~~~~~~~~~~~~~~~~~~~~~~~~~~~~~~~~~~~~~~~~~~~~~~~~~~~~~~~~~
!| fformat        |-->| format of the file
!| fid            |-->| file descriptor
!| nvar           |<->| number of variable
!| ierr           |<--| 0 if no error during the opening
!~~~~~~~~~~~~~~~~~~~~~~~~~~~~~~~~~~~~~~~~~~~~~~~~~~~~~~~~~~~~~~~~~~~~~~~
!
  character(len=8), intent(in)  :: fformat
  integer,          intent(in)  :: fid
  integer,          intent(inout) :: nvar
  integer,          intent(out) :: ierr
!
end subroutine
\end{lstlisting}
%
\begin{lstlisting}
subroutine get_data_var_list (fformat,fid,nvar,varlist,unitlist,ierr)
!
!brief    returns a list of all the name of the variables in the mesh file
!+        and a list of their units
!
!~~~~~~~~~~~~~~~~~~~~~~~~~~~~~~~~~~~~~~~~~~~~~~~~~~~~~~~~~~~~~~~~~~~~~~~
!| fformat        |-->| format of the file
!| fid            |-->| file descriptor
!| varlist        |<->| list of variable name
!| untilist       |<->| list of variable unit
!| ierr           |<--| 0 if no error during the execution
!~~~~~~~~~~~~~~~~~~~~~~~~~~~~~~~~~~~~~~~~~~~~~~~~~~~~~~~~~~~~~~~~~~~~~~~
!
  character(len=8),  intent(in)  :: fformat
  integer,           intent(in)  :: fid
  integer,           intent(in)  :: nvar
  character(len=16), intent(inout) :: varlist(nvar)
  character(len=16), intent(inout) :: unitlist(nvar)
  integer,           intent(out) :: ierr
!
end subroutine
\end{lstlisting}
%
\begin{lstlisting}
subroutine get_data_ntimestep (fformat,fid,ntimestep,ierr)
!
!brief    returns the number of time step in the mesh file
!
!~~~~~~~~~~~~~~~~~~~~~~~~~~~~~~~~~~~~~~~~~~~~~~~~~~~~~~~~~~~~~~~~~~~~~~~
!| fformat        |-->| format of the file
!| fid            |-->| file descriptor
!| ntimestep      |<->| the number of time steps
!| ierr           |<--| 0 if no error during the execution
!~~~~~~~~~~~~~~~~~~~~~~~~~~~~~~~~~~~~~~~~~~~~~~~~~~~~~~~~~~~~~~~~~~~~~~~
!
  character(len=8), intent(in)  :: fformat
  integer,          intent(in)  :: fid
  integer,          intent(inout) :: ntimestep
  integer,          intent(out) :: ierr
!
end subroutine
\end{lstlisting}
%
\begin{lstlisting}
subroutine get_data_time (fformat,fid,record,time,ierr)
!
!brief    returns the time value of a given time step
!
!~~~~~~~~~~~~~~~~~~~~~~~~~~~~~~~~~~~~~~~~~~~~~~~~~~~~~~~~~~~~~~~~~~~~~~~
!| fformat        |-->| format of the file
!| fid            |-->| file descriptor
!| record         |-->| number of the time step
!| time           |<->| time in second of the time step
!| ierr           |<--| 0 if no error during the execution
!~~~~~~~~~~~~~~~~~~~~~~~~~~~~~~~~~~~~~~~~~~~~~~~~~~~~~~~~~~~~~~~~~~~~~~~
!
  character(len=8), intent(in)  :: fformat
  integer,          intent(in)  :: fid
  integer,          intent(in)  :: record
  double precision, intent(inout) :: time
  integer,          intent(out) :: ierr
!
end subroutine
\end{lstlisting}
%
\begin{lstlisting}
subroutine get_data_value (fformat,fid,record,var_name,res_value,n,ierr)
!
!brief    returns the value for each point of a given variable 
!+        for a given time step
!
!~~~~~~~~~~~~~~~~~~~~~~~~~~~~~~~~~~~~~~~~~~~~~~~~~~~~~~~~~~~~~~~~~~~~~~~
!| fformat        |-->| format of the file
!| file_id        |-->| file descriptor
!| record         |-->| time step to read in the file
!| var_name       |-->| variable for which we need the value
!| res_value      |<->| value for each point at time step record 
!|                |   | for the variable var_name
!| n              |-->| size of res_value
!| ierr           |<--| 0 if no error during the execution
!~~~~~~~~~~~~~~~~~~~~~~~~~~~~~~~~~~~~~~~~~~~~~~~~~~~~~~~~~~~~~~~~~~~~~~~
!
  character(len=8),  intent(in)    :: fformat
  integer,           intent(in)    :: fid
  integer,           intent(in)    :: record
  character(len=16), intent(in)    :: var_name
  integer,           intent(in)    :: n
  double precision,  intent(inout) :: res_value(n)
  integer,           intent(out)   :: ierr
!
end subroutine
\end{lstlisting}
%
\subsection{Writing functions}
%
\begin{lstlisting}
subroutine set_header (fformat,file_id,title,nvar,var_name,ierr)
!
!brief    writes the title and the name and units of the variables
!
!~~~~~~~~~~~~~~~~~~~~~~~~~~~~~~~~~~~~~~~~~~~~~~~~~~~~~~~~~~~~~~~~~~~~~~~
!| fformat        |-->| format of the file
!| file_id        |-->| file descriptor
!| title          |-->| title of the mesh
!| nvar           |-->| number of variables
!| var_name       |-->| name and units of the variables
!| ierr           |<--| 0 if no error during the opening
!~~~~~~~~~~~~~~~~~~~~~~~~~~~~~~~~~~~~~~~~~~~~~~~~~~~~~~~~~~~~~~~~~~~~~~~
!
  character(len=8),  intent(in)  :: fformat
  integer,           intent(in)  :: file_id
  character(len=80), intent(in)  :: title
  integer,           intent(in)  :: nvar
  character(len=32), intent(in)  :: var_name(nvar)
  integer,           intent(out) :: ierr
!
end subroutine
\end{lstlisting}
%
\begin{lstlisting}
subroutine set_mesh
     (fformat,file_id,mesh_dim,typelm,ndp,nptfr,
      nptir,nelem,npoin,ikle,ipobo,
      knolg,x,y,nplan,date,time,ierr,z)
!
!brief    writes the mesh geometry in the file
!
!~~~~~~~~~~~~~~~~~~~~~~~~~~~~~~~~~~~~~~~~~~~~~~~~~~~~~~~~~~~~~~~~~~~~~~~
!| fformat        |-->| format of the file
!| file_id        |-->| file descriptor
!| mesh_dim       |-->| dimension of the mesh
!| typelm         |-->| type of the mesh elements
!| ndp            |-->| number of points per element
!| nptfr          |-->| number of boundary point
!| nptir          |-->| number of interface point
!| nelem          |-->| number of element in the mesh
!| npoin          |-->| number of points in the mesh
!| ikle           |-->| connectivity array for the main element
!| ipobo          |-->| is a boundary point ? array
!| knolg          |-->| local to global numbering array
!| x              |-->| x coordinates of the mesh points
!| y              |-->| y coordinates of the mesh points
!| nplan          |-->| number of planes
!| date           |-->| date of the creation of the mesh
!| time           |-->| time of the creation of the mesh
!| ierr           |<--| 0 if no error during the opening
!| z  (optional)  |-->| z coordinates of the mesh points
!~~~~~~~~~~~~~~~~~~~~~~~~~~~~~~~~~~~~~~~~~~~~~~~~~~~~~~~~~~~~~~~~~~~~~~~
!
  character(len=8), intent(in)  :: fformat
  integer,          intent(in)  :: file_id,nplan
  integer,          intent(in)  :: date(3)
  integer,          intent(in)  :: time(3)
  integer,          intent(in)  :: mesh_dim
  integer,          intent(in)  :: typelm
  integer,          intent(in)  :: ndp
  integer,          intent(in)  :: nptfr
  integer,          intent(in)  :: nptir
  integer,          intent(in)  :: nelem
  integer,          intent(in)  :: npoin
  integer,          intent(in)  :: ikle(nelem*ndp)
  integer,          intent(in)  :: ipobo(npoin)
  integer,          intent(in)  :: knolg(npoin)
  double precision, intent(in)  :: x(npoin),y(npoin)
  integer,          intent(out) :: ierr
  double precision, intent(in), optional :: z(npoin)
!
end subroutine
\end{lstlisting}
%
\begin{lstlisting}
subroutine set_bnd (fformat,fid,type_bnd_elt,nelebd,ndp,ikle,value,ierr)
!
!brief    writes the boundary information into the mesh file
!
!~~~~~~~~~~~~~~~~~~~~~~~~~~~~~~~~~~~~~~~~~~~~~~~~~~~~~~~~~~~~~~~~~~~~~~~
!| fformat        |-->| format of the file
!| fid            |-->| file descriptor
!| type_bnd_elt   |-->| type of the boundary elements
!| nelebd         |-->| number of boundary elements
!| ndp            |-->| number of points per boundary element
!| ikle           |-->| connectivity array for the boundary elements
!| value          |-->| value for each boundary element
!| ierr           |<--| 0 if no error during the opening
!~~~~~~~~~~~~~~~~~~~~~~~~~~~~~~~~~~~~~~~~~~~~~~~~~~~~~~~~~~~~~~~~~~~~~~~
!
  character(len=8), intent(in)  :: fformat
  integer,          intent(in)  :: fid
  integer,          intent(in)  :: type_bnd_elt
  integer,          intent(in)  :: nelebd
  integer,          intent(in)  :: ndp
  integer,          intent(in)  :: ikle(nelebd*ndp)
  integer,          intent(in)  :: value(nelebd)
  integer,          intent(out) :: ierr
!
end subroutine
\end{lstlisting}
%
\begin{lstlisting}
subroutine add_data
     (fformat,file_id,var_name,time,record,
      first_var,var_value,n,ierr)
!
!brief    add data information for a given variable and a given time on 
!+        all points of the mesh
!
!~~~~~~~~~~~~~~~~~~~~~~~~~~~~~~~~~~~~~~~~~~~~~~~~~~~~~~~~~~~~~~~~~~~~~~~
!| fformat        |-->| format of the file
!| file_id        |-->| file descriptor
!| var_name       |-->| name of the variable
!| time           |-->| time of the data
!| record         |-->| time step of the data
!| first_var      |-->| true if it is the first variable of the dataset
!| var_value      |-->| the value for each point of the mesh
!| n              |-->| size of var_value
!| ierr           |<--| 0 if no error during the opening
!~~~~~~~~~~~~~~~~~~~~~~~~~~~~~~~~~~~~~~~~~~~~~~~~~~~~~~~~~~~~~~~~~~~~~~~
!
      character(len=8),  intent(in)  :: fformat
      integer,           intent(in)  :: file_id,n
      character(len=32), intent(in)  :: var_name
      double precision,  intent(in)  :: time
      integer,           intent(in)  :: record
      logical,           intent(in)  :: first_var
      double precision,  intent(in)  :: var_value(n)
      integer,           intent(out) :: ierr
!
end subroutine
\end{lstlisting}

%
%--------------------------------------------------------------------------------
\chapter{Developer Manual}
%--------------------------------------------------------------------------------
%

%
%--------------------------------------------------------------------------------
\section{Structure of the module}
%--------------------------------------------------------------------------------
%
The module is divided into three parts:
\begin{itemize}
\item The file \verb!interface_hermes.f! which contains the list of the
reading/writing functions.
\item A file for each of the function listed in \verb!interface_hermes.f!.
\item A module for each format handled in \telemacsystem.
\end{itemize}
%
All the files of the second part look more or less like this:
\begin{lstlisting}
!                    ********************
                     SUBROUTINE OPEN_MESH
!                    ********************
!
     &(FFORMAT,FILE_NAME,FILE_ID,OPENMODE,IERR)
!
!***********************************************************************
! HERMES   V7P0                                               01/05/2014
!***********************************************************************
!
!brief    OPENS A MESH FILE
!
!history  Y AUDOUIN (LNHE)
!+        24/03/2014
!+        V7P0
!+
!
!~~~~~~~~~~~~~~~~~~~~~~~~~~~~~~~~~~~~~~~~~~~~~~~~~~~~~~~~~~~~~~~~~~~~~~~
!| FFORMAT        |-->| FORMAT OF THE FILE
!| FILE_NAME      |-->| Name of the file
!| FILE_ID        |-->| File descriptor
!| OPENMODE       |-->| ONE OF THE FOLLOWING VALUE 'READ','WRITE','READWRITE'
!| Ierr           |<--| 0 if no error during the opening
!~~~~~~~~~~~~~~~~~~~~~~~~~~~~~~~~~~~~~~~~~~~~~~~~~~~~~~~~~~~~~~~~~~~~~~~
!
      USE UTILS_SERAFIN
      USE UTILS_MED
      IMPLICIT NONE
      INTEGER     LNG,LU
      COMMON/INFO/LNG,LU
!
!+-+-+-+-+-+-+-+-+-+-+-+-+-+-+-+-+-+-+-+-+-+-+-+-+-+-+-+-+-+-+-+-+-+-+-+
!
      CHARACTER(LEN=8),  INTENT(IN)    :: FFORMAT
      CHARACTER(LEN=*), INTENT(IN)   :: FILE_NAME
      INTEGER,            INTENT(IN)   :: FILE_ID
      CHARACTER(LEN=9), INTENT(IN)     :: OPENMODE
      INTEGER, INTENT(OUT)             :: IERR
!
!+-+-+-+-+-+-+-+-+-+-+-+-+-+-+-+-+-+-+-+-+-+-+-+-+-+-+-+-+-+-+-+-+-+-+-+
!
      SELECT CASE (FFORMAT)
        CASE ('SERAFIN ','SERAFIND')
          CALL OPEN_MESH_SRF(FILE_NAME, FILE_ID, OPENMODE,FFORMAT, IERR)
        CASE ('MED     ')
          CALL OPEN_MESH_MED(FILE_NAME, FILE_ID, OPENMODE,IERR)
        CASE DEFAULT
          IF(LNG.EQ.1) THEN
            WRITE(LU,*) 'OPEN_MESH : MAUVAIS FORMAT : ',FFORMAT
          ENDIF
          IF(LNG.EQ.2) THEN
            WRITE(LU,*) 'OPEN_MESH: BAD FILE FORMAT: ',FFORMAT
          ENDIF
          CALL PLANTE(1)
          STOP
      END SELECT
!
!-----------------------------------------------------------------------
!
      RETURN
      END
\end{lstlisting}
%
Their goal is just to test the format value and redirect to the
format-specified implementation of the function.\\
%
The modules of the third part contain the implementation of those format
specific functions.\\
%
Each file is associated to an id which is associated to a Fortran custom-type
object defined in the format module. This object contains any information
necessary to running Hermes with the format (for example: the number of nodes,
elements\ldots).



%
%--------------------------------------------------------------------------------
\section{Serafin Format}
%--------------------------------------------------------------------------------
%

%
\subsection{Format description}
%
The SERAFIN format is a Format Binary file (Each record is written in-between
two tags, 4-bytes-integers, containing the size of the record) composed of
the list of following records:
\begin{itemize}
\item TITLE: title of the mesh (80 characters long)
\item NBV1,NBV2: Number of variables: with a linear discretisation, with a
quadratic discretisation, NBV2 is now obsolete its value is 0.
\item NBV1+NBV2 records:
\begin{itemize}
\item NAME AND UNIT OF A VARIABLE (32 characters long)
\end{itemize}
\item 1,0,0,0,0,0,NPLAN,NPTFR,NPTIR,HAS\_DATE: NPLAN is the number of planes
this is only in a 3D mesh, NPTFR and NPTIR are only present in a partitioned
file they are respectively the number of boundary points and the number of
interface points
\item If HAS\_DATE is not equal to 0 then the following record is in the
file:\\ YEAR,MONTH,DAY,HOUR,MINUTE,SECOND: date of the creation of the file
\item NELEM, NPOIN, NDP, 1: Number of elements, Number of points, number of
points per element
\item IKLE: connectivity table: array of dimension (NDP,NELEM) whereas in
\telemacsystem the dimensions are (NELEM,NDP)
\item IPOBO/KNOLG: array of integer dimension NPOIN, IPOBO contains 0 for the
internal points and it provides a numbering of the boundary points for the
others. If the file is partitioned (i.e. parallel run of the code) instead it
is called KNOLG and it contains the local to global numbering of the mesh
points
\item X array of real dimension NPOIN, contains the X coordinates of the mesh
points
\item Y array of real dimension NPOIN, contains the Y coordinates of the mesh
points
\item The following are then found for each time step:
\begin{itemize}
\item TIME: a real, time value for the time step
\item Values for variable 1 at time TIME, dimension NPOIN
\item \ldots
\item Values for variable NBV1+NBV2 at time TIME, dimension NPOIN
\end{itemize}
\end{itemize}

%
\subsection{New functionalities in Hermes}
%
To allow the Hermes functions to be generic, the way to read/write SERAFIN
format was modified. Before Hermes, a normal read/write structure was used,
because of that there was a lot of calls to the "rewind" functions and
constraints on how to read/write. Especially when reading the data informations
as the file was read sequentially (i.e. always starting for the begin of the
file and having to skip things to get to the data you wanted).\\
%
To change that the "stream" access was used instead, it allows to read the file
as a C binary file (i.e. it treats the file as a continuous sequence of bytes,
addressable by a positive integer starting from 1). By building a map when
opening the file we can have "direct" access to the data needed.\\
%
Here below is the object used for the SERAFIN format the first part is the map
of the file and the second one some value from the file that are frequently
needed so that the file is not reread for those.\\
%
A couple of fixed values are also set such as the size of single/double
precision for integer and real as well as the size of string (title, variable
name and unit...).
%
\begin{lstlisting} 
integer, parameter :: is = 4 ! integer size
integer, parameter :: i4 = selected_int_kind(5) ! integer size
integer, parameter :: i8 = selected_int_kind(15) ! integer on 8 bytes size
integer, parameter :: r4 = selected_real_kind(5) ! single precision size
integer, parameter :: r8 = selected_real_kind(15) ! double precision size
integer, parameter :: var_size = 32 ! size of a variable text
integer, parameter :: title_size = 80 ! size of a title
!
type srf_info
  ! size of elements
  integer :: rs ! real size (4 or 8)
  ! position in file
  integer :: pos_title 
  integer :: pos_nvar != pos_title + 4 + title_size + 4
  integer :: pos_varinfo != pos_nvar + 4 + 2*is + 4
  integer :: pos_ib != pos_varinfo + 4 + nvar*var_size + 4
  integer :: pos_date != pos_ib + 4 + 10*is + 4
  integer :: pos_num != pos_date + (ib(10).ne.0)*(4 + 6*is + 4)
  integer :: pos_ikle != pos_num + 4 + 4*is + 4 
  integer :: pos_ipobo != pos_ikle + 4 + nelem*ndp*is + 4
  integer :: pos_coord != pos_ipobo + 4 + npoin*is + 4
  integer :: pos_data != pos_coord + (4 + npoin*rs + 4)*ndim
  ! computed informations
  integer :: size_data != 4 + npoin*rs + 4
  integer :: size_data_set != 4 + rs + 4 + nvar*(4 + npoin*rs + 4)
  ! stocked quantities and small variables
  integer :: ntimestep
  integer :: npoin
  integer :: nvar
  integer :: nelem
  integer :: ndp
  integer :: nplan
  integer :: nptir
  integer :: ndim
  integer :: typ_elt
  character(len=var_size),allocatable :: var_list(:) 
  ! boundary informations
  integer :: typ_bnd_elt
  integer :: nptfr
  integer :: ncli
end type srf_info
\end{lstlisting}
%
The \verb!RS! sets the size of the real value (4 for single precision SERAFIN
format, 8 for double precision SERAFIND format).
%
%--------------------------------------------------------------------------------
\section{MED format}
%--------------------------------------------------------------------------------
%
\subsection{Format description}
%
The goal of the MED format (EDF and CEA, L-GPL licence) is to standardize
exchange of data between scientific computational codes. For instance, this
format is used by the SALOME platform (\url{http://www.salome-platform.org})
to manage transfer between the different modules. The MED format is a compact
binary storage of meshes, results and computational data based on the HDF5
library. The generic storage structure of the MED format offers to store
several meshes, fields, profiles, groups at different times in the same file.
The mesh can be structured or not, based on different elements and can change
in time. For HPC aspects, reading and writing can be executed in parallel in
the single file. More information can be found in the MED documentation
(\url{http://www.code-aster.org/outils/med/html/index.html}).
%
\subsection{New functionalities in Hermes}
%
To adapt to \telemacsystem a structure was defined for the boundary file to work with
MED. The group structure (which allows to define a list of points/elements as a
group) available in MED was used. The MED boundary file follows this structure:
\begin{itemize}
\item NGROUP: Number of groups
\item LIHBOR,LIUBOR,LIVBOR,LITBOR,GROUP: (Type of boundary on h,u,v,tracer and
the name of the group)
\end{itemize}
%
Here is a simple example:
\begin{verbatim}
3
2 2 2 2 BOUNDARY1
5 4 4 4 BOUNDARY2
4 5 5 5 BOUNDARY3
\end{verbatim}

% 
%--------------------------------------------------------------------------------
\section{How to add a new format}
%--------------------------------------------------------------------------------
%
To add a new format (named \textit{format}) the following steps must be fulfilled:
\begin{itemize}
\item Create the file "utils\_\textit{format}.f" and implement all the functions
described in \ref{listfunc}
\item Update all the functions generic file to add a new branch to the "if"
statement for \textit{format}
\item Same thing in \verb!bief\_open\_files.f! and \verb!bief_close_files.f!
\item Add in all the dictionary the option to choose \textit{format} for the
"FORMAT" keywords 
\end{itemize}

%
%--------------------------------------------------------------------------------
\section{Error Handling}
%--------------------------------------------------------------------------------
%
A few error values where created in \verb!declarations_special.f!:
\begin{itemize}
\item \verb!HERMES_RECORD_UNKNOWN_ERR!
\item \verb!HERMES_VAR_UNKNOWN_ERR!
\item \verb!HERMES_FILE_ID_ALREADY_IN_USE_ERR!
\item \verb!HERMES_FILE_NOT_OPENED_ERR!
\item \verb!HERMES_MAX_FILE_ERR!
\item \verb!HERMES_WRONG_ARRAY_SIZE_ERR!
\item \verb!HERMES_MED_NOT_LOADED_ERR!
\item \verb!HERMES_UNKNOWN_ELEMENT_TYPE_ERR!
\item \verb!HERMES_WRONG_ELEMENT_TYPE_ERR!
\item \verb!HERMES_UNKNOWN_GROUP_ERR!
\item \verb!HERMES_WRONG_HDF_FORMAT_ERR!
\item \verb!HERMES_WRONG_MED_FORMAT_ERR!
\item \verb!HERMES_WRONG_MED_VERSION_ERR!
\item \verb!HERMES_WRONG_AXE_ERR!
\end{itemize}


%
%--------------------------------------------------------------------------------
\section{Future work}
%--------------------------------------------------------------------------------
%
Here is a list of modifications that could be interesting in the future:
\begin{itemize}
\item Switch to boundary on elements in Hermes and in \telemacsystem
\item Add 3D mesh evolving with time (MED only), pending development in
ParaVis to handle that
\item Use MED possibility to have multiple meshes for coupling
\end{itemize}
