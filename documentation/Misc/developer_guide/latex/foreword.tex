%
\chapter{Foreword}
%
%
This is the lastest release of the developer guide to the \telemacsystem, based
on FORTRAN 2003. It has been written to help the numerous people who have to
develop or to understand the "ins" and "outs" of this system, namely research
engineers and technicians at EDF, students and researchers in universities,
research institutes and laboratories, or users willing to write specific user
subroutines. It will probably not meet all the expectations: giving fully
detailed explanations on all the system would take thousands of pages, and
would probably never be read! With this guide we only hope to establish a
closer relationship between developers, and we shall enhance the guide
progressively, as new questions arise. This document will be a success if you
consider it yours. We thus beg you to report on errors, misprints and mistakes,
and to ask for more explanations on parts that would not be clear enough. It
will be a commitment for us to take into account all your remarks in next
releases.
%
%
\chapter{Structure of this guide}
%
%
This guide is made of four main parts and a number of appendices. Chapter
\ref{ref:programmingbief} should be the only useful one for developers of
programs based on the \bief library.  Chapter \ref{ref:internalbief} give
details on the very structure of \bief and is \textit{a priori} meant for \bief
developers themselves. Chapter \ref{ref:devlife} is about the different program
a developer will have to use. Chapter \ref{codingconv} is the coding convention
that should be followed when developping in \telemacsystem.
