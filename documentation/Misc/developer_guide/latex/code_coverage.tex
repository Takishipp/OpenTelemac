%
%%%%%%%%%%%%%%%%%%%%%%%%%%%%%%%%%%%%%%%%%%%%%%%%%%%%%%%%%%%%%%%%%%%%%%%%
\chapter{Code coverage}
%%%%%%%%%%%%%%%%%%%%%%%%%%%%%%%%%%%%%%%%%%%%%%%%%%%%%%%%%%%%%%%%%%%%%%%%
%
This chapter will explain how to run the code coverage of the code using gcov and lcov.
%
%
\section{What is Code Coverage}
%
%
Dixit Wikipedia:"In computer science, code coverage is a measure used to
describe the degree to which the source code of a program is tested by a
particular test suite. A program with high code coverage, measured as a
percentage, has had more of its source code executed during testing which
suggests it has a lower chance of containing undetected software bugs compared
to a program with low code coverage.Many different metrics can be used to
calculate code coverage; some of the most basic are the percent of program
subroutines and the percent of program statements called during execution of
the test suite."
%
%
\section{What to do}
%
%
You can find all the information for the configuration in systel.edf.cfg with
the configuration gcov.

Then run the following script:
\begin{lstlisting}[language=bash]
#!/bin/bash
# First run to set counter to zero
lcov --directory $HOMETEL/builds/$USETELCFG/lib --capture \
--initial --output-file $HOMETEL/app.info
# Running test cases
validateTELEMAC.py -k2 --clean --bypass
# Gathering data
lcov --directory $HOMETEL/builds/$USETELCFG/lib --capture \
--output-file $HOMETEL/app.info
# Generating html output
genhtml --legend --highlight \
--output-directory $HOMETEL/documentation/code_coverage \
-t "Telemac-Mascaret V&V code coverage" $HOMETEL/app.info
\end{lstlisting}

This will build the html display under
\verb!<root>/documentation/code_coverage!. That display will contains for each
folder of the sources directory the percentage of code/functions used and if
you follow through you can even see the file and exactly what line was used.
