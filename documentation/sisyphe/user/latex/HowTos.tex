%-------------------------------------------------------------------------------
\chapter[How to]{How to?}
%-------------------------------------------------------------------------------
\pagebreak
%-------------------------------------------------------------------------------
\section{Compute sediment fluxes through a given section(s)}
%-------------------------------------------------------------------------------
Use the keywords {\ttfamily FLUXLINE} (logical type, set to {\ttfamily NON} by default) and {\ttfamily FLUXLINE INPUT FILE} (character type).

%\pagebreak
%-------------------------------------------------------------------------------
\section{Implement a new bedload transport formula}
%-------------------------------------------------------------------------------
To implement a new bedload transport formula, the keyword {\ttfamily BED-LOAD TRANSPORT FORMULA} must be set to {\ttfamily = 0}. The Fortran subroutine must be added into the fortran file of \textsc{Telemac-2d} or \textsc{Telemac-3d}, keyword {\ttfamily FORTRAN FILE}.

The template subroutine is called \texttt{qsfrom.f} and can be found in the folder \texttt{/sources/sisyphe}
\begin{lstlisting}[frame=trBL]
!                    ***************** 
                     SUBROUTINE QSFORM 
!                    ***************** 
     &(U2D, V2D, TOB, HN, XMVE, TETAP, MU, NPOIN, DM,  
     & DENS, GRAV, DSTAR, AC, QSC, QSS) 
! 
!***********************************************************************
! SISYPHE   V6P2                                   21/07/2011 
!***********************************************************************
! 
!brief    ALLOWS THE USER TO CODE THEIR OWN BEDLOAD TRANSPORT 
!+                FORMULATION, BEST SUITED TO THEIR APPLICATION. 
! 
!~~~~~~~~~~~~~~~~~~~~~~~~~~~~~~~~~~~~~~~~~~~~~~~~~~~~~~~~~~~~~~~~~~~~~~~
!~~~~~~~~~~~~~~~~~~~~~~~~~~~~~~~~~~~~~~~~~~~~~~~~~~~~~~~~~~~~~~~~~~~~~~~
! 
      USE INTERFACE_SISYPHE, EX_QSFORM => QSFORM 
!     USE DECLARATIONS_SISYPHE 
      USE BIEF 
      IMPLICIT NONE 
      INTEGER LNG,LU 
      COMMON/INFO/LNG,LU 
! 
!+-+-+-+-+-+-+-+-+-+-+-+-+-+-+-+-+-+-+-+-+-+-+-+-+-+-+-+-+-+-+-+-+-+-+-+
! 
      TYPE(BIEF_OBJ),   INTENT(IN)    :: U2D,V2D,TOB,HN,TETAP,MU 
      TYPE(BIEF_OBJ),   INTENT(INOUT) :: QSC, QSS 
      INTEGER,          INTENT(IN)    :: NPOIN 
      DOUBLE PRECISION, INTENT(IN)    :: XMVE, DM, DENS, GRAV, DSTAR, AC
! 
!+-+-+-+-+-+-+-+-+-+-+-+-+-+-+-+-+-+-+-+-+-+-+-+-+-+-+-+-+-+-+-+-+-+-+-+
! 
! 
      INTEGER          :: I 
      DOUBLE PRECISION :: C1, C2, T 
      DOUBLE PRECISION, PARAMETER :: ACOEFF = 0.004D0!Sediment transport param (m^2s^-1)
! 
!======================================================================!
!======================================================================!
!                               PROGRAM                                !
!======================================================================!
!======================================================================!
! 
!     GRASS (1981) TYPE 
!      
      DO I = 1, NPOIN 
 
        QSC%R(I) = ACOEFF * U2D%R(I) * (U2D%R(I)**2+V2D%R(I)**2) ! 1D Grass (1981)  
        QSS%R(I) = 0.D0                                          ! Zero suspended load
 
      END DO 
! 
! 
!-----------------------------------------------------------------------
! 
      RETURN 
      END
\end{lstlisting}      

%\pagebreak
%-------------------------------------------------------------------------------
\section{Define a rigid bed}
%-------------------------------------------------------------------------------
\subsection{Data from selafin file}
\subsection{Data coded by the user in the fortran file}

%\pagebreak
%-------------------------------------------------------------------------------
\section{Print a new output variable in the selafin file}
%-------------------------------------------------------------------------------
\begin{itemize}
\item Declare the {\ttfamily PRIVE} variable, for example as:
  
{\ttfamily USE DECLARATIONS\_SISYPHE, ONLY : PRIVE}

\item Use the following expression to include the variable you want to visualize:

{\ttfamily PRIVE\%ADR(N)\%P\%R(K) = [Here the variable you want to visualize]}, where {\ttfamily N} is the number of variables that you want to visualize and {\ttfamily K} is the number of nodes.  

\item In the \sisyphe's steering file you can use the flags {\ttfamily'A'} or {\ttfamily'G'} to visualize the {\ttfamily PRIVE} variable, for example as:
  
{\ttfamily VARIABLES FOR GRAPHIC PRINTOUTS='U,V,S,H,B,Q,M,E,QSBL,TOB,MU,A'}
\end{itemize}

The default name {\ttfamily PRIVE 1} (for {\ttfamily N=1}) can be modified in the subroutine {\ttfamily nomvar\_sisyphe.f}.

\begin{lstlisting}[frame=trBL]
DO K=1, NPOIN
  PRIVE%ADR(1)%P%R(K) = [variable to visualize]
ENDDO  
\end{lstlisting}  

%\pagebreak
%-------------------------------------------------------------------------------
\section{Introduce a new keyword}
%-------------------------------------------------------------------------------
\begin{itemize}
\item In {\ttfamily declarations\_sisyphe.f} declare the variable to be called from a keyword e.g. {\ttfamily HMIN\_BEDLOAD}
\item In {\ttfamily lecdon\_sisyphe.f} declare .... {\ttfamily HMIN\_BEDLOAD=MOTREA(ADRESS(2,52))}
\item Declaration in the modified subroutine through {\ttfamily USE DECLARATIONS\_SISYPHE, ONLY : HMIN\_BEDLOAD}
\end{itemize}


%\pagebreak
%-------------------------------------------------------------------------------
\section{Read and use a variable from a selafin file}
%-------------------------------------------------------------------------------
Case of spatially distributed sediment zones

\begin{itemize}
\item Create the different zones with, e.g. BlueKenue
\item Add in your steering file:
  \begin{lstlisting}[frame=trBL]
NUMBER OF PRIVATE ARRAYS = 1
NAMES OF PRIVATE VARIABLES= 'ZONE                            '
NAMES OF PRIVATE VARIABLES= 'ZONE____________________________'
\end{lstlisting}
(32 characters)\textvisiblespace\textvisiblespace\textvisiblespace
\item add
\begin{lstlisting}[frame=trBL]
      DO J=1,NPOIN
      NCOUCHES(J)=1
	  IF(PRIVE%ADR(1)%P%R(J).EQ.1.D0) THEN
	  AVAIL(J,1,1)=1.D0
	  AVAIL(J,1,2)=0.D0
	  ELSEIF(PRIVE%ADR(1)%P%R(J).EQ.2.D0) THEN
	  AVAIL(J,1,1)=0.D0
	  AVAIL(J,1,2)=1.D0	 
	  ELSE 
	  AVAIL(J,1,1)=0.5D0
	  AVAIL(J,1,2)=0.5D0
	  ENDIF
      ENDDO
\end{lstlisting}
\end{itemize}


%\pagebreak
%-------------------------------------------------------------------------------
\section{Define the soil stratigraphy (init\_compo)}
%-------------------------------------------------------------------------------

%\pagebreak
%-------------------------------------------------------------------------------
\section{Suppress bed updating}
%-------------------------------------------------------------------------------
Set \telkey{STATIONARY MODE = YES} (logical type, set to {\ttfamily = NO} by default)           


%\pagebreak
%-------------------------------------------------------------------------------
\section{Using a non-declared variable in a Sisyphe's subroutine}
%-------------------------------------------------------------------------------
If you want to use, for example, parameter \texttt{NPTFR} and the table \texttt{NBOR(NPTFR)} in the subroutine NOEROD,
declare:

\texttt{USE DECLARATIONS\_SISYPHE, ONLY : NPTFR, MESH}

\texttt{INTEGER, POINTER :: NBOR(:)}

Then the following alias can be declared:

\texttt{NBOR=>MESH\%NBOR\%I}
