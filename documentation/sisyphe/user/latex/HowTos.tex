%-------------------------------------------------------------------------------
\chapter[How to]{How to?}
%-------------------------------------------------------------------------------
\pagebreak
%-------------------------------------------------------------------------------
\section{Compute sediment fluxes through a given section(s)}
%-------------------------------------------------------------------------------
Use the keywords {\ttfamily FLUXLINE} (logical type, set to {\ttfamily NON} by default) and {\ttfamily FLUXLINE INPUT FILE} (character type).

\pagebreak
%-------------------------------------------------------------------------------
\section{Implement a new bedload transport formula}
%-------------------------------------------------------------------------------
To implement a new bedload transport formula, the keyword {\ttfamily BED-LOAD TRANSPORT FORMULA} must be set to {\ttfamily = 0}. The Fortran subroutine must be added into the fortran file of \textsc{Telemac-2d} or \textsc{Telemac-3d}, keyword {\ttfamily FORTRAN FILE}.

The template subroutine is called \texttt{qsfrom.f} and can be found in the folder \texttt{/sources/sisyphe}
\begin{lstlisting}[frame=trBL]
!                    ***************** 
                     SUBROUTINE QSFORM 
!                    ***************** 
     &(U2D, V2D, TOB, HN, XMVE, TETAP, MU, NPOIN, DM,  
     & DENS, GRAV, DSTAR, AC, QSC, QSS) 
! 
!***********************************************************************
! SISYPHE   V6P2                                   21/07/2011 
!***********************************************************************
! 
!brief    ALLOWS THE USER TO CODE THEIR OWN BEDLOAD TRANSPORT 
!+                FORMULATION, BEST SUITED TO THEIR APPLICATION. 
! 
!~~~~~~~~~~~~~~~~~~~~~~~~~~~~~~~~~~~~~~~~~~~~~~~~~~~~~~~~~~~~~~~~~~~~~~~
!~~~~~~~~~~~~~~~~~~~~~~~~~~~~~~~~~~~~~~~~~~~~~~~~~~~~~~~~~~~~~~~~~~~~~~~
! 
      USE INTERFACE_SISYPHE, EX_QSFORM => QSFORM 
!     USE DECLARATIONS_SISYPHE 
      USE BIEF 
      IMPLICIT NONE 
      INTEGER LNG,LU 
      COMMON/INFO/LNG,LU 
! 
!+-+-+-+-+-+-+-+-+-+-+-+-+-+-+-+-+-+-+-+-+-+-+-+-+-+-+-+-+-+-+-+-+-+-+-+
! 
      TYPE(BIEF_OBJ),   INTENT(IN)    :: U2D,V2D,TOB,HN,TETAP,MU 
      TYPE(BIEF_OBJ),   INTENT(INOUT) :: QSC, QSS 
      INTEGER,          INTENT(IN)    :: NPOIN 
      DOUBLE PRECISION, INTENT(IN)    :: XMVE, DM, DENS, GRAV, DSTAR, AC
! 
!+-+-+-+-+-+-+-+-+-+-+-+-+-+-+-+-+-+-+-+-+-+-+-+-+-+-+-+-+-+-+-+-+-+-+-+
! 
! 
      INTEGER          :: I 
      DOUBLE PRECISION :: C1, C2, T 
      DOUBLE PRECISION, PARAMETER :: ACOEFF = 0.004D0!Sediment transport param (m^2s^-1)
! 
!======================================================================!
!======================================================================!
!                               PROGRAM                                !
!======================================================================!
!======================================================================!
! 
!     GRASS (1981) TYPE 
!      
      DO I = 1, NPOIN 
 
        QSC%R(I) = ACOEFF * U2D%R(I) * (U2D%R(I)**2+V2D%R(I)**2) ! 1D Grass (1981)  
        QSS%R(I) = 0.D0                                          ! Zero suspended load
 
      END DO 
! 
! 
!-----------------------------------------------------------------------
! 
      RETURN 
      END
\end{lstlisting}      

%\pagebreak
%-------------------------------------------------------------------------------
%\section{Define a rigid bed}
%-------------------------------------------------------------------------------
%\subsection{Data from selafin file}
%\subsection{Data coded by the user in the fortran file}

%\pagebreak
%-------------------------------------------------------------------------------
%\section{Print a new output variable in the selafin file}
%-------------------------------------------------------------------------------

%\pagebreak
%-------------------------------------------------------------------------------
%\section{Introduce a new keyword}
%-------------------------------------------------------------------------------

%\pagebreak
%-------------------------------------------------------------------------------
%\section{Read and use a variable from a selafin file}
%-------------------------------------------------------------------------------

%\pagebreak
%-------------------------------------------------------------------------------
%\section{Define the soil stratigraphy (init\_compo)}
%-------------------------------------------------------------------------------

%\pagebreak
%-------------------------------------------------------------------------------
%\section{Suppress bed updating}
%-------------------------------------------------------------------------------
%\telkey{STATIONARY MODE} (logical type, set to {\ttfamily = NO} by default)           
