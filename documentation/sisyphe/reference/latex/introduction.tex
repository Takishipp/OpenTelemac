\section{Introduction}\label{sec:introduction}
\sisyphe is a sediment transport and morphodynamic simulation module which is part of the hydroinformatic finite elements and finite volume system {\scshape Telemac-Mascaret}. In this module, sediment transport rates, split into bedload and suspended load, are calculated at each grid point as a function of various flow (velocity, water depth, wave height, etc.) and sediment (grain diameter, relative density, settling velocity, etc.) parameters. The bedload is calculated by using classical sediment transport formulae from the literature. The suspended load is determined by solving an additional transport equation for the depth-averaged suspended sediment concentration. The bed evolution equation (Exner equation) can be solved by using either a finite element or a finite volume formulation.

\sisyphe is applicable to non-cohesive sediments (uniform or graded), cohesive sediments as well as to sand-mud mixtures. The sediment composition is represented by a finite number of classes, each characterized by its mean diameter, grain density and settling velocity. Sediment transport processes can also include the effect of bottom slope, rigid beds, secondary currents, slope failure, etc. For cohesive sediments, the effect of bed consolidation can be accounted for.

\sisyphe can be applied to a large variety of hydrodynamic conditions including rivers, estuaries and coastal applications, where the effects of waves superimposed to a tidal current can be included. The bed shear stress, decomposed into skin friction and form drag, can be calculated either by imposing a friction coefficient (Strickler, Nikuradse, Manning, Ch\'{e}zy or user defined) or by a bed-roughness predictor.

\subsection{Coupling with hydrodynamics}
In \sisyphe the relevant hydrodynamic variables can be either imposed in the model (chaining method) or calculated by a hydrodynamic computation (internal coupling). It is convenient to use one of the hydrodynamic modules of the \tel system (\teldd, \telddd or \tomawac) for compatibility reasons (same mesh, same pre- and post-processor, etc.), but the user can also choose a different hydrodynamic model. The different methods which can be used to prescribe the hydrodynamics are described in \S \ref{sec:running:subsec:coupling}.

\subsection{Numerical kernel}
\sisyphe can be run on Unix, Linux or Windows. The latest release of \sisyphe (version 6.2) uses the version 6.2 of the {\scshape Bief} finite element library (\tel system library). 

\subsection{Preprocessing and Postprocessing}
We refer the reader to the specific documentation for pre- and post-processing tools, see e.g. \cite{}.

\subsection{Outline of the manual}
The main steps to run a sedimentological computation are given in Section \ref{sec:running}. 
In Section \ref{sec:interactions}, a description of the main sediment and hydrodynamics parameters such that total bed shear stress, skin friction, etc. is presented. 
In Section \ref{ch:bedload} the bedload is presented. In Section \ref{ch:suspension}, the suspended load transport is introduced. Section \ref{ch:waves} presents the influence of the effect of waves on the sediment transport and bed evolution. In Section \ref{ch:sandgrading}, the sand grading is introduced.
%cohesive sediment. Section \ref{ch:9} presents the numerical schemes and methods to reduce CPU time, for long term morphodynamics simulations (of the order of decades) and
%medium scale basins. Section \ref{ch:10} a brief discussion on the morphodynamic model accuracy and limitations is presented.
