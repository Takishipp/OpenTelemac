\section{Mixed sediments}\footnote{This chapter has been written by D. Phan van Bang, Lan and Villaret}

\label{ch:mixed}

\subsection{Sediment bed composition}

Characteristics of each class\newline
Sand-mud mixture can be represented in SISYPHE by using two classes of bed
sediments (NSICLA= 2). This recent developments available in release 6.2
require further testing and improvement. \newline
The first class (noted 1) is non-cohesive and represented by its grain
diameter D$_{1}$ and can be transported as bed-load or suspended load. The
settling velocity W$_{s1}$ is a function of the relative sediment density
(s=1.65) and grain diameter D$_{1}$. So far we assume only suspended load,
which implies that the model is devoted to mixtures of fine sand grains and
mud.\newline
The second class (noted 2) is cohesive , grain diameter D$_{2}$ less than 60
mm. The settling velocity W$_{s2}$ is a function of flocs properties which
differs from the individual cohesive particles, and needs to be specified. 
\newline
Vertical structure of the sediment bed\newline
The vertical structure can be stratified as for pure mud. The bed is
discretized into vertical layers up to a maximum number of layers (NOMBLAY%
\texttt{<}20). Each layer is characterized by a constant value of the mass
concentration for the mud, which can be specified in the steering file (C$%
_{si}$ for i=1, NOMBLAY). \newline
Note: the mass concentration of the mud component is defined for the mud
only (i.e. mass of sediments per volume of mud). The initial bed layer
thicknesses can be specified in subroutine init\_mixte.f, and varies as the
bed undergoes erosion/deposition. The consolidation of sand/mud mixture is
not yet developed. \newline
The time and spatial variation in the sediment composition is obtained by
variation in the composition of each layers (Percent of the mud and mass of
sand per layers). \newline
Percent of mud/sand mixture\newline
For layer j\newline

Phase 1: Sand grains, density $\rho$$_{s}$

Phase 2: Mud bed mass density C$_{s}$

Es

The composition of the sediment bed depends on the percent of each class (P$%
_{1}$,P$_{2}$) needs to be specified. In consistency with the algorithm for
sand grading effects, P$_{i}$ represents the volume of class i divided by
the total volume, such that $P_{2} +P_{1} =1$

\begin{equation*}
P_{i} =\frac{V_{i} }{V_{t} } 
\end{equation*}
where V$_{i}$ is the volume occupied by component i, and $V_{t} =V_{2}
+V_{1} $, the total volume.\newline
The total thickness of layer j is decomposed into Es= Es$_{1}$ + Es$_{2}$
with Es$_{i}$ = p$_{i}$ Es.

The total mass of the sand-mud mixture (M$_{t}$) is :

\begin{equation*}
M_{t} =C_{s} V_{2} +\rho _{s} V_{1} =\left( C_{s} P_{2} +\rho _{s} P_{1}
\right) V_{t} 
\end{equation*}
With $\rho$$_{s}$ = 2650 Kg/m$^{3}$ (sand density) which is constant, while
for the mud, C$_{s}$ depends on the consolidation state and is constant per
sediment bed layer. It is specified by using keyword 'MUD CONCENTRATION PER
LAYER'.\newline
Mass balance\newline
The total mass of the sand-mud mixture (M$_{t}$) is :

\begin{equation*}
M_{t} =C_{s} V_{2} +\rho _{s} V_{1} =\left( C_{s} P_{2} +\rho _{s} P_{1}
\right) V_{t} 
\end{equation*}
We define for each class the mass per surface area and per layer:

\begin{gather*}
MS\_VASE=CsP_{2} Es \\
MS\_SAND=\rho _{s} P_{1} Es
\end{gather*}

For the mass balance of the mud phase (phase 2)

\begin{equation*}
M_{2} =\sum\limits_{i=1,Npoin}\left[ \underbrace{\sum%
\limits_{j=1,Nomblay}P_{2}^{} Cs_{}^{} Es_{} } Ms\_vase\right] _{j} S_{i} 
\end{equation*}%
Initialization\newline
The initial percent of each class (p$_{1}$,p$_{2}$) needs to be specified.
The keyword `INITIAL FRACTION FOR PARTICULAR SIZE CLASS' can be applied, if
the initial distribution is constant (per layer and per node). \newline
In the mass balance for each class, the total mass per unit surface area
must account for the fraction of each class:

Keywords:\newline
Type of sediments\newline
$\neg$\hspace{5mm} 'MIXED SEDIMENT' (default MIXTE = No)\newline
$\neg$\hspace{5mm} If Mixte =Yes: \newline
o\hspace{5mm} 'COHESIVE SEDIMENTS' (SEDCO) =YES, NO\hspace{5mm} \newline
o\hspace{5mm} 'NUMBER OF SIZE-CLASSES OF BED MATERIAL' (NSICLA) = 2\newline
$\neg$\hspace{5mm} `SEDIMENT DIAMETERS' = D1\texttt{>}0.00006 (non-cohesive
sediment), D2(D$_{2}$\texttt{<}0.00006 m, for cohesive sediment)\newline
$\neg$\hspace{5mm} `SEDIMENT DENSITY' ($\rho$$_{s}$= 2650 Kg/m$^{3}$,
default value)\newline
$\neg$\hspace{5mm} `settling velocities' = Ws$_{1}$, Ws$_{2}$

Bed composition\newline
$\neg$\hspace{5mm} 'NUMBER OF LAYERS OF THE CONSOLIDATION MODEL'
(NOMBLAY=10, default )'\newline
$\neg$\hspace{5mm} 'MUD CONCENTRATION PER LAYER' in Kg/m$^{3}$ (CONC\_VASE=
50.; 100.;..,by default )\newline
$\neg$\hspace{5mm} INITIAL FRACTION FOR PARTICULAR SIZE CLASS : p$_{1}$; p$%
_{2}$

\subsection{Erosion/deposition fluxes}

The erosion/deposition fluxes now depends on the mass \% of each class in
the surface layer (Panagiotopoulos et al., 1997). If the mass percent of the
mud class is greater than 50\% the bed is considered as pure `cohesive' and
the erosion/deposition laws follow the Partheniades classical law. (see
below for the calculation of the bed shear strength in a sand/mud mixture).%
\newline
If the mud percentage is less than 30 \%, the bed is considered as non
cohesive, and has little effect on the bed shear strength.\newline
Once sediment particles have been put in suspension, they are transported
independently, by solving for each class, a transport equation for the
volume concentration of the individual sediment particles (deflocculated)
defined as:

\begin{equation*}
C_{\substack{ i  \\ }}^{} =\frac{C_{i} }{\rho _{s} } 
\end{equation*}

Mud flocs in suspension

Sand grains in suspension

\begin{equation*}
\frac{\partial C_{i} }{\partial t} +U_{conv} \frac{\partial C_{i} }{\partial
x} +V_{conv} \frac{\partial C_{i} }{\partial y} =\left[ \frac{\partial }{%
\partial x} \left( \epsilon _{s} \frac{\partial C_{i} }{\partial x} \right) +%
\frac{\partial }{\partial y} \left( \epsilon _{s} \frac{\partial C_{i} }{%
\partial y} \right) \right] +\frac{(E_{i} -D_{i} )}{h} 
\end{equation*}%
The erosion flux is determined for each class of sediments as a function of
the bed composition.\newline
The deposition flux D= W$_{si}$ C$_{i}$ is an implicit term.

Erosion law for sand mud mixture

The rate of erosion is based on the Partheniades erosion law, where the
critical bed shear strength of the mud class depends on the consolidation
state.

The bed shear strength of the sand mud mixtures depends on the \% of mud (P$%
_{2}$ at the surface top layer). We follow here the method of Waeles (2005,
[36]).\newline
The erosion rate E$_{(1+2)}$ is calculated for the mixture and then for each
class Ei.

E$_{i}$= P$_{i}$ E$_{(1+2)}$

$\neg$ C\texttt{>}50\%: mud dominant\newline
We apply the Krone erosion law:

\begin{gather*}
E_{(1+2)} =M\left[ \left( \frac{u_{*} }{u_{*e}^{} } \right) ^{2} -1\right]
\;\;for\;\;\tau _{b} =\rho u_{*}^{2} >\tau _{ce} =\rho u_{*e}^{2} \\
E_{(1+2)} =0\;\;if\,\tau _{b} <\tau _{ce}
\end{gather*}

$\neg$ C\texttt{<}30\%: sand dominant\newline
We apply the Zyserman and Fredsoe equilibrium concentration:

\begin{equation*}
E_{(1+2)} =W_{s1} C_{eq} \;\;for\;\;\tau _{b} =\rho u_{*}^{2} >\tau _{ce}
=\rho u_{*e}^{2} 
\end{equation*}
$\neg$ 30\%\texttt{<}C\texttt{<}50\%: intermediate range.\newline
We assume a linear interpolation of erosion rates for each class.

\subsection{Bed evolution}

The bed evolution for both phases is calculated differently following the
method for pure mud or pure sand; This is not entierely correct and should
be modified in the near future:

For sand (phase 1).\newline
We assume the first layer to be greater than the active layer thickness,
such that the sand percent can be considered to be constant and equal to the
percent of the top layer.

\begin{equation*}
P_{1} dZ_{f1} =(D_{1} -E_{1} )Dt 
\end{equation*}
In this mass balance for sand, we do not consider the void ratio, since it
is entirely filled by the fine sediments (the volume concentration of the
sand phase is 1, instead of (n-1), where n is the bed porosity in case of
either pure sand or sand mixtures).\newline
If there is net erosion $(E_{1} -D_{1} )>0$: the first layer Es$_{1}$ is
decreased Es$_{1}$=Es$_{1}$-P$_{1}$dZf$_{1}$\newline
which is only correct if Es$_{1}$-P$_{1}$dZf$_{1}$\texttt{>}0$_{.}$\newline
If there is net deposition $(D_{1} -E_{1} )>0$: the sediment is deposited in
the first top layer Es1 is increased Es$_{1}$=Es$_{1}$-P1dZf$_{1}$

For mud (phase 2)\newline
We follow the method described in part. The top layers are successively
eroded until we match the erosion flux.\newline
In case of deposition $(D_{2} -E_{2} )>0$: the sediment is deposited in the
top layer

\begin{equation*}
P_2 C s_2 \,DZf_{f_2} =\rho_s (D_2 - E_2) Dt 
\end{equation*}
The top layer thickness is increased in order to match : Es$_{2}$= Es$_{2}$
+ dZf$_{2}$\newline
In case of erosion the procedure consist in determining the maximum layer to
be eroded

\begin{equation*}
\sum\limits_{j=1}^{j\max -1}Ms_{2} (j) <\rho _{s} (E_{2} -D_{2}
)Dt<\sum\limits_{j=1}^{j\max }Ms_{2} (j) 
\end{equation*}
All top layers (up to jmax-1) are emptied{\nobreakspace}: \newline
For the last one, the mass balance is now

\begin{equation*}
\sum\limits_{j=1}^{j\max -1}Ms_{2} (j) +dEs_{2} (j\max )Cs(j\max )=\rho _{s}
(E_{2} -D_{2} )Dt 
\end{equation*}
Reactualisation of the bed composition\newline
The percent of each class of sediment needs to be recalculated (end of
suspension main).

\begin{equation*}
P_{i} =\frac{Es_{i} }{Es} 
\end{equation*}%
I

With Es= Es$_{1}$+Es$_{2}$\newline


%\section*{ Conclusions}

%This report presents recent developments in SISYPHE 6.2 for the treatment of
%cohesive sediment transport (for erosion, deposition, bed evolution and
%consolidation algorithm). Preliminary features for mixed sediments are also
%described, although this part is still under development and definitely
%needs further validation based on well documented data sets. The test cases
%for cohesive sediments are based on experiments on the Gironde mud and
%further applications under in-situ conditions can be found in Van's PhD
%thesis (2012).\newline

\subsection*{References}

Pham Van Bang D., Lefran\c{c}ois E., Sergent P., Bertrand F., 2008. MRI
experimental and finite elements modelling of the
sedimentation-consolidation of mud, La Houille Blanche, 168, n$^\circ$%
3-2008, 39-44.\newline
Sanchez, M., 1992, Mod\'{e}lisation dans un estuaire \`{a} mare: R\^{o}le du
bouchon vaseux dans la tenue des sols sous marins. PhD thesis. University of
Nantes (232 pages).\newline
Bartholomeeusen, G., Sills, G.C., Znidarcic, D., van Kesteren, W.,
Merckelbach, L.M., Pyke, R., Carrier, W.D., Lin, H., Penumadu, D.,
Winterwerp, H., Masala, S. and Chan, D., 2002, Sidere: numerical prediction
of large strain consolidation, G\'{e}otechnique 52, No. 9, 639-648.\newline
Been, K. and Sills, G.C., 1981, Self weight consolidation of Soft Soils: an
Experimental and Theoretical Study. Geotechnique, Volume 31, No.4, pp.
519-535.\newline
Camenen B. \& Pham Van Bang D., 2011. Modelling the settling of suspended
sediments for concentrations close to the gelling concentration, Cont. Shelf
Res., 31, S106-S116.\newline
Hervouet, J.M. 2007. Hydrodynamics of free surface flow, modelling with
finite elements system. Wiley. 341p.\newline
Gibson, R.E., Englund, G. L. and Hussey, M. J. L., 1967, The theory of one
dimensional consolidation of saturated clay, I. Finite Non-Linear
Consolidation of Thin Homogeneous Layers. Geotechnique, pp. 261-273\newline
Gibson, R.E., Schiffman, R.L. and Cargill, K.W., 1981, The theory of One
dimensional Consolidation of Saturated Clays, II. Finite Nonlinear
Consolidation of Thick Homogeneous Layers, Canadian Geotechnical Journal,
Vol. 18, pp. 280-293\newline
Hervouet J.M. (2007): Hydrodynamics of Free Surface Flows, modelling with
the finite- element method, J. Wiley \& Sons Ltd, West Sussex, England, 340pp%
\newline
Lenormant, C., Lepeintre, F., Teisson, C., Malcherek, A., Markofsky, M., and
Zielke, W., 1993, Three dimensional modelling of estuarine processes. In
MAST Days and Euromar Market, Project Reports Volume 1.\newline
Leveque, Randal J., 2002, Finite-Volime Methods for Hyperbolics Problems,
Cambridge Texts in Applied Mathematics, ISBN 0-511-04219-1 eBook.\newline
Migniot C., 1968. A study of the physical properties of different very fine
sediments and their behaviour under hydrodynamic action, La Houille Blanche
7, 591-620.\newline
Partheniades, E., 1962, A study of erosion and deposition of cohesive soils
in salt water. Ph.D thesis, University of California, Berkeley, 182 p.

Pham Van Bang D., Ovarlez G., Tocquer L., 2007. Density and structural
effects on the rheological characteristics of mud, La Houille Blanche, 2,
85-93.

Tassi, P., Villaret C., Huybrechts,N. Hervouet, JM. (2011): Numerical
modelling of 2D and 3D suspended sediment transport in turbulent flows,
Proceedings of the RCEM 2011 Conference in Beijing.\newline
Pham Van Bang D., Lefran\c{c}ois E., Sergent P., Bertrand F., 2008. MRI
experimental and finite elements modelling of the
sedimentation-consolidation of mud, La Houille Blanche, 168, n$^\circ$%
3-2008, 39-44.\newline
Sanchez, M., 1992, Mod\'{e}lisation dans un estuaire \`{a} mare: R\^{o}le du
bouchon vaseux dans la tenue des sols sous marins. PhD thesis. University of
Nantes (232 pages).\newline
Terzaghi, K. 1923. Die Berechnung des Durchlassigkeitsziffer des Tones aus
des Verlauf des hydrodynamischen Spannungerscheinungen, Sitz. Akad. Wissen.
Wien, Math. Naturwiss. Kl., Abt IIa, 132, 125-138.\newline
Thi\'{e}bot J., 2008, Mod\'{e}lisation num\'{e}rique des processus
gouvernant la formation et la d\'{e}gradation des massif vaseux (PhD thesis
ENGREF- U.Caen), 130 p. \newline
Thiebot, J., Guillou, S., Brun-Cottan, J-C., 2011, An optimisation method
for determining permeability and effective stress relationships of
consolidating cohesive sediment deposits, Continental Shelf Research 31
(2011) S117-S123.\newline
Toorman E. A., 1996, Sedimentation and self-weight consolidation: general
unifying theory, G\'{e}otechnique 46, N$^\circ$ 1, pp. 101-113\newline
Toorman, E.A., 1999. Sedimentation and self-weight consolidation:
constitutive equations and numerical modelling. G\'{e}otechnique,
49(6):709-726.

Van L.A., 2012: Mod\'{e}lisation du transport des sediments mixtes
sable-vase et application \`{a} la morphodynamique de l'estuaire de la
Gironde, PhD Thesis, Universit\'{e} Paris-Est.\newline
Van L.A., Villaret C., Pham van Bang D., Sch\"{u}ttrumpf H., 2012, Erosion
and deposition of the Gironde mud, International Conference on Scour and
Erosion, ICSE-6 -- Paris- August 27-31, 2012.\newline
Van L.A., Pham Van Bang D., 2012, Hindered settling of sand/mud flocs
mixtures: from model formulation to numerical validation, Advances in Water
Resources (accepted).\newline
Van Leussen, 1994: Estuarine macroflocs and their role in fine grained
transport, PhD Thesis, Utrecht University(Nederlands).

Villaret, C., Tassi, P. 2012{\nobreakspace}: SISYPHE User manual, release
6.2,. Rapport EDF-LNHE H-P74-2012-02004-EN .\newline
Villaret C., Walther R. ,2008: Numerical modeling of the Gironde estuary.
Physics of Estuaries and Coastal Sediments, Liverpool, August 2008.\newline
Villaret C., Van L.A., Huybrechts N., Pham Van Bang D., Boucher O. 2010.
Consolidation effects on morphodynamics modelling: application to the
Gironde estuary, La Houille Blanche, N$^\circ$ 6-2010, 15-24.\newline
Winterwerp, J.C. and van Kesteren, W.G.M., 2004, Introduction to the physics
of cohesive sediment in the marine environment, Development in Sedimentology
56, Elsevier (466 pages).

Waeles B., 2005: Detachment and transport of slay sand gravel mixtures by
channel flows, Ph. D. Thesis University of Caen.



