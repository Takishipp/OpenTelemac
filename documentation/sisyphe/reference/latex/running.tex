\section{Running a sedimentological computation}\label{sec:running}

\subsection{Input files}
The minimum set of input files to run a \sisyphe simulation includes the steering file (text file {\sffamily cas}), the geometry file (format selafin {\sffamily slf}), and the boundary conditions file (text file {\sffamily cli}). Additional or optional input files include the fortran file, the reference file, the result file, etc.

\subsection{Steering file}
The steering file contains the necessary information for running a computation, plus the values of parameters
that are different from the defaults.
Like in other modules of the \tel system, the model parameters can
be specified in the (obligatory) \sisyphe steering file. The following essential information (input/output) should be specified in the steering file:
\begin{itemize}
\item Input and output files 
\item Physical parameters (sand diameter, cohesive or
not, settling velocity, etc.)
\item Main sediment transport processes (transport mechanism and formulae, etc.)
\item Additional sediment transport processes (secondary currents, slope effect, etc.)
\item Numerical options and parameters (numerical scheme, options for solvers, etc.)
\end{itemize}

\subsection{Coupling hydrodynamics and morphodynamics}\label{sec:running:subsec:coupling} 
We describe here two methods for linking the hydrodynamic and the
morphodynamic models: by \emph{chaining} (the flow is obtained from a previous
hydrodynamic simulation assuming a fixed bed) or by \emph{internal coupling} (both the
flow and bed evolution are updated at each time step).

\subsubsection{Chaining method}
\begin{itemize}
\item \textbf{Principle}

Both models (hydrodynamic and morphodynamic) are run independently: during the first hydrodynamic simulation 
the bed is assumed to be fixed. Then, in the subsequent morphodynamic run, 
the flow rate and free surface are read from the previous hydrodynamic results file.
% ***START COMPLETE***
This `chaining method' is only justified for relatively simple flows, due to
the difference in time-scales between the hydrodynamics and the bed
evolution. For unsteady tidal flow, \sisyphe can be used in an unsteady mode:
the flow field is linearly interpolated between two time steps of the
hydrodynamic file. For steady flow, the last time step of the hydrodynamic
file is used and the flow rate and free surface assumed to stay constant
while the bed evolves. 
\\

% ***END COMPLETE***

\item \textbf{Flow updating}

At each time step, the flow velocity is updated by assuming simply that both
the flow rate and the free surface elevation are conserved, such that, in
the case of deposition, the flow velocity is locally increased, whereas in
the case of erosion, the flow velocity decreases.

This rather schematic updating does not take into account any deviation of
the flow. It is only suitable for simple flows (2D processes) and assuming
relatively small bed evolutions. However it can be responsible for numerical
instabilities~\cite{}.%ADD (cf. [01b], Hervouet and Villaret, 2004).
% ***START VERIFY***
The morphodynamic computation is stopped when the bed evolution reaches a
certain percent of the initial water depth. This simple updating of the flow
field is no longer valid when the bed evolution becomes greater than a
significant percentage of the water depth, specified by the user. At this
point, it is recommended to stop the morphodynamic calculation and to
recalculate the hydrodynamic variables. 
% ***END VERIFY***
\\

\item \textbf{Mass continuity}

It should be noted that with this simple method, the sediment mass
continuity may not be satisfied because of potential losses due to changes
in the flow depth as the bed evolves. 

When the flow is steady ({\ttfamily STEADY CASE = YES}), only the last record of the previous result file will be
used. Otherwise ({\ttfamily STEADY CASE = NO}), the {\ttfamily TIDE PERIOD} and {\ttfamily NUMBER OF TIDES OR FLOODS} will be used to specify the sequence to be read on the
hydrodynamic files. Hydrodynamic records are interpolated at each time step
of the sedimentological computation.

Note: an error may occur when the {\ttfamily TIDE PERIOD} is not a multiple of the
graphical time steps of the hydrodynamic file ({\ttfamily hydrodynamic file is not
long enough}). In an unsteady case, the keyword {\ttfamily STARTING TIME OF THE HYDROGRAM} gives the
first time step to be read. If the starting time is not specified, the last
period of the hydrogram will be used for sedimentological computation.
\\

\item \textbf{Steering/fortran files}

For uncoupled mode, the \sisyphe steering file should specify:
\begin{itemize}
\item The time steps, graphical or listing output,
duration 
\item The hydrodynamic file as yielded by \teldd or \telddd ({\ttfamily HYDRODYNAMIC FILE}) or by the subroutine {\ttfamily condim\_sisyphe.f}. 
\item For waves only: the wave parameters can be either
calculated by a wave propagation code (\tomawac), or defined directly in
\sisyphe ({\ttfamily condim\_sisyphe.f}). The effect of waves on bed forms and associated
bed roughness coefficient can be accounted with keyword: {\ttfamily EFFECT OF WAVES = YES}.
\item A restart from a previous \sisyphe model run, by
setting {\ttfamily COMPUTATION CONTINUED = YES} and specification of sedimentological
results in {\ttfamily PREVIOUS SEDIMENTOLOGICAL COMPUTATION}
\item Flow options: steady or unsteady options, flow
period
\end{itemize}

\medskip
\begin{bclogo}[couleur=blue!10,arrondi=0.1, logo=\bcinfo]{Keywords}
For time step, duration and output:
\begin{itemize}
\item {\ttfamily TIME STEP, NUMBER OF TIME STEPS}
\item {\ttfamily GRAPHIC PRINTOUT PERIOD} 
\item {\ttfamily LISTING VARIABLES FOR GRAPHIC PRINTOUTS}
\end{itemize}
For hydrodynamics (imposed flow and updated):
\begin{itemize}
\item {\ttfamily HYDRODYNAMIC FILE}
\item {\ttfamily STEADY CASE =NO}, default option
\item {\ttfamily TIDE PERIOD = 44640}, default option
\item {\ttfamily STARTING TIME OF THE HYDROGRAM = 0.}, default
option
\item {\ttfamily NUMBER OF TIDES OR FLOODS = 1}, default option
\item {\ttfamily CRITICAL EVOLUTION RATIO = 0.1}, default value
\end{itemize}
For waves: 
\begin{itemize}
\item {\ttfamily WAVE FILE, WAVE EFFECTS}
\end{itemize}
\end{bclogo}

\end{itemize}

\subsubsection{Internal coupling}
\begin{itemize}
\item \textbf{Principle}

\sisyphe can be automatically coupled
(internally) with the hydrodynamic model, \teldd or \telddd. \sisyphe is
called inside the hydrodynamic model without any exchange of data files. The
data to be exchanged between the two programs is now directly shared in the
memory, instead of being written and read in a file. %FOR CHAINING METHOD????

At each time step, the hydrodynamics variables (velocity field, water depth,
bed shear stress,...) are transferred to the morphodynamic model, which
sends back the updated bed elevation to the hydrodynamics model.
\\

\item \textbf{Time step and coupling period}
The internal coupling method is more CPU time consuming than the chaining
method. Various techniques can be set up to reduce the CPU time (e.g.
parallel processors). 
% ***START VERIFY***
In certain cases, the use of a coupling
period $>1$ allows the bed load transport rates and resulting bed
evolution not to be re-calculated at every time step.
% ***END VERIFY***
For suspended load, a diffusion/transport equation needs to be solved.
This transport equation obeys the same Courant number criteria on the time
step than the hydrodynamics, and therefore needs to be solved at each
time-step ({\ttfamily COUPLING PERIOD = 1}).

The time step of \sisyphe is equal to the time step of \teldd or \telddd
multiplied by the `coupling period'. The graphic/listing printout periods
are the same as in the \tel computation.

The \teldd/3D steering file must specify the type of coupling, the name
of the \sisyphe steering file, and the coupling period. In addition, the
Fortran file of \sisyphe must be sometimes specified in the \tel steering
file (if there is no Fortran file for \tel). Some of the keywords of the
\sisyphe steering file become obsolete.

\medskip
\begin{bclogo}[couleur=blue!10,arrondi=0.1, logo=\bcinfo]{Keywords}
For internal coupling, the following keywords need to be specified in the
\teldd or \telddd steering files:
\begin{itemize}
\item {\ttfamily COUPLING WITH = SISYPHE} 
\item {\ttfamily COUPLING PERIOD =1}, default value 
\item {\ttfamily NAME OF SISYPHE STEERING FILE}
\end{itemize}
\end{bclogo}

All computational parameters (time step, duration, printout, option for
friction) need to be specified in the \tel steering file, but are no
longer used by \sisyphe. The values of time step, bottom shear stress, etc. are
transferred directly from \tel to \sisyphe.
% EXPLAIN BETTER!!!!
\begin{bclogo}[couleur = blue!10, arrondi = 0.10, logo = \bcattention]{\textsf{Important}}
The following keywords are no longer in use is (\sisyphe) steering file:
\begin{itemize}
\item {\ttfamily TIME STEP}
\item {\ttfamily GRAPHIC PRINTOUT PERIOD}
\item {\ttfamily LISTING PRINTOUT PERIOD}
\item {\ttfamily LAW OF BOTTOM FRICTION}
\item {\ttfamily FRICTION COEFFICIENT}
\end{itemize}
\end{bclogo}

\end{itemize}


%\subsection{Input data files}
%The geometry file is a binary file that contains the finite element mesh information (e.g., element type, number of elements, coordinates of the nodes and connectivity matrix). 
%This file can also contain bottom topography information and/or friction values at each mesh point. 

%The geometry file ({\ttfamily GEOMETRY FILE}) is a binary file in Selafin format issued
%from the mesh generator. It contains a full description
%of the (three-node) triangular grid as well as the bathymetry information, obtained from the interpolation of bathymetric data file onto each node of the triangular grid.

%The {\ttfamily FORTRAN FILE} contains those subroutines which the user has modified.

%The {\ttfamily HYDRODYNAMIC FILE} contains the hydrodynamic results of a previous
%hydrodynamic computation (\teldd or \telddd). If \tel is coupled to \sisyphe, the grid must be the same. 
%When a hydrodynamics file comes for example from a finite differences hydrodynamic model, 
%the file must be transformed beforehand and interpolated onto a triangular, finite element mesh.

%The hydrodynamics will be given either by the last time step of that file in
%steady mode, or by the time steps related to the period of the tide or flood
%being considered in unsteady mode. That file should contain current data and wave data if the effect of waves is considered. The {\ttfamily WAVE FILE} contains the hydrodynamic results of a previous wave
%computation (\tomawac). Only the last record will be read if the keyword {\ttfamily WAVE EFFECTS} is
%activated.

%The {\ttfamily PREVIOUS SEDIMENTOLOGICAL COMPUTATION FILE} is provided for implementing
%a computational sequence, it contains the results of a previous
%sedimentological computation made using \sisyphe on the same grid. If that
%file exists, then the initial sedimentological conditions of the computation
%will be given by the last time step in the file. This file will be read
%only if the keyword {\ttfamily COMPUTATION CONTINUED} is activated.

%The {\ttfamily REFERENCE FILE} is the results file from a previous computation. The last time step is compared to the new last time step. 
%All the files must be either binary (Selafin format) or text ({\ttfamily ascii} format). 

%\medskip
%\begin{bclogo}[couleur=blue!10,arrondi=0.1, logo=\bcinfo]{Keywords}
%\begin{itemize}
%\item {\ttfamily BOUNDARY CONDITIONS FILE}
%\item {\ttfamily GEOMETRY FILE}
%\item {\ttfamily FORTRAN FILE}
%\item {\ttfamily RESULTS FILE}
%\end{itemize}
%When combining with a previous computation:
%\begin{itemize}
%\item {\ttfamily COMPUTATION CONTINUED} {\ttfamily = NO}, default option
%\item {\ttfamily PREVIOUS SEDIMENTOLOGICAL COMPUTATION FILE}
%\end{itemize}
%\end{bclogo}

\subsection{Boundary conditions file}
The format of the boundary condition file is the same as for \teldd or \telddd. 
This file can be created by a mesh generator (for example, BlueKenue~\cite{bluekenue}) and modified using a text editor. 
Each line is related to a point along the edge of the mesh. Boundary points are listed in the file in the following way: 
First the domain outline points, proceeding counterclockwise from the lower left corner, then the islands proceeding clockwise, also from the lower left corner. 

The edge points are numbered like the file lines; the numbering first
describes the domain outline in the counterclockwise direction from lower
left point (X+Y minimum), then the islands in the clockwise direction. The
boundary condition about the bottom depth is imposed at the specific place
of the tracer boundary condition in \teldd.
The following thirteen variables for each edge point are first read out of
the boundary conditions file \texttt{X1, X2, X3, X4, X5, X6, X7, LIEBOR, EBOR, X10,
X11, N, K}. The first seven variables (\texttt{X1, X2, X3, X4, X5, X6, X7}), as well as \texttt{X10} and
\texttt{X11}, are specific to the \teldd model. They are read in Sisyphe, but
remain unused.

