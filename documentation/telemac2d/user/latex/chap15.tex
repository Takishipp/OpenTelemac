

\chapter{ Parallelism}

 TELEMAC-2D is generally run on single-processor computers of the workstation type. When simulations call for high-capacity computers and in the absence of a super-computer, it may be useful to run the computations on multi-processor (or multi-core) computers or clusters of workstations. A parallel version of TELEMAC-2D is available for use with this type of computer architecture.

 The parallel version of TELEMAC-2D uses the MPIMPI library, which must therefore be installed beforehand. The interface between TELEMAC-2D and the MPI library is the ``parallel''parallel library library common to all modules of the TELEMAC system (in folder /sources/utils/parallel).

 Informations on the use of the parallel version is given in the system installation documents.

 Initially, the user must specify the number of processors used by means of the keyword \textit{PARALLEL PROCESSORS}. The keyword may have the following values:

\begin{enumerate}
\item [\nonumber] 0: Use of the classical version of TELEMAC-2D

\item [\nonumber] 1: Use of the parallel version of TELEMAC-2D with one processor

\item [\nonumber] N: Use of the parallel version of TELEMAC-2D with the specified number of processors, here N (it can work also just for testing on a single processor!).
\end{enumerate}

 Domain decomposition and results file combination operations are now automatic and handled completely by the start-up procedure.

 Parallel machines are eventually configured by a single file (see system installation document).

 Note that for python version of TELEMAC, number of processors is given as an argument for the launching command and not as a hard-coded keyword in the steering file (e.g. telemac2d.py --ncsize=4 cas.txt will run TELEMAC-2D on 4 processors).







