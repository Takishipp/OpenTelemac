%==================================
%==================================
\section{Introduction}
%==================================
%==================================
%==================================
\subsection{A word of caution}
%==================================
This document contains information about the quality of a complex modelling tool. Its purpose is to assist the user in assessing the reliability and accuracy of computational results, and to provide guidelines with respect to the applicability and judicious employment of this tool. This document does not, however, provide mathematical proof of the correctness of results for a specific application. The reader is referred to the License Agreement for pertinent legal terms and conditions associated with the use of the software.

The contents of this validation document attest to the fact that computational modelling of complex physical systems requires great care and inherently involves a number of uncertain factors. In order to obtain useful and accurate results for a particular application, the use of high-quality modelling tools is necessary but not sufficient. Ultimately, the quality of the computational results that can be achieved will depend upon the adequacy of available data as well as a suitable choice of model and modelling parameters.
% 
\subsection{Document plan}
%==================================

%==================================
\subsection{Validation layout}
% %==================================

This validation is presented hereafter using a \textit{validation sheet form},
each sheet detailing the physical concepts involved, the physical and numerical parameters used and comparing both numerical and reference solutions.
Then, each sheet displays the following informations:
\begin{list}{-}{}
\item [-] \textbf{Purpose \& Problem description} : These first two parts give reader short details about the test case, the physical phenomena involved and specify how the numerical solution will be validated;
\item [-] \textbf{Reference} : This part gives the reference solution we are comparing to and explicits the analytical solution when available;
\item [-] \textbf{Physical parameters} : This part specifies the geometry,
details all the physical parameters used to describe both porous media (soil model in particularly) and
solute characteristics (dispersion/diffusion coefficients, soil $\equiv$ pollutant interactions...);
\item [-] \textbf{Geometry and Mesh} : This part describes the mesh used in the \tomawac computation;
\item [-] \textbf{Initial and boundary conditions} : this part details both initial and boundary conditions used to simulate the case ;
\item [-] \textbf{Numerical parameters} : this part is used to specify the numerical parameters used
(adaptive time step, mass-lumping when necessary...);
\item [-] \textbf{Results} : we comment in this part the numerical results against the reference ones,
giving understanding keys and making assumptions when necessary.
\end{list}
%
\bigskip
%
\clearpage
%==================================
%==================================
\section{Presentation}
%==================================
%==================================
\subsection{General}
%==================================
TOMAWAC is a scientific software which models the changes, both in the time and in the spatial domain, of the power spectrum of wind-driven waves and wave agitation for applications in the oceanic domain, in the intracontinental seas as well as in the coastal zone. The model uses the finite elements formalism for discretizing the sea domain; it is based on the computational subroutines of the TELEMAC system as developed by the EDF R\&D’s Laboratoire National d'Hydraulique et Environnement (LNHE). TOMAWAC is one of the models making up the TELEMAC system 
The acronym TOMAWAC being adopted for naming the software was derived from the following English denomination:

TELEMAC-based Operational Model Addressing Wave Action Computation

TOMAWAC can be used for three types of applications:
\begin{itemize}
\item	Wave climate forecasting a few days ahead, from wind field forecasts. This real time type of application is rather directed to weather-forecasting institutes such as Météo-France, whose one mission consists of predicting continuously the weather developments and, as the case may be, publishing storm warnings.
\item	Hindcasting of exceptional events having severely damaged maritime structures and for which field records are either incomplete or unavailable.
\item	Study of wave climatology and maritime or coastal site features, through the application of various, medium or extreme, weather conditions in order to obtain the conditions necessary to carry out projects and studies (harbour constructions, morphodynamic coastal evolutions, ...).
\end{itemize}

%==================================
\subsection{Capabilities}
%==================================
 \subsubsection{Application domain of the model TOMAWAC}
\label{par31}
TOMAWAC is designed to be applied from the ocean domain up to the coastal zone. The limits of the application range can be determined by the value of the relative depth d/L, wherein d denotes the water height (in metres) and L denotes the wave length (in metres) corresponding to the peak spectral frequency for irregular waves.

The application domain of TOMAWAC includes:
\begin{itemize}
\item {\bf the oceanic domain}, characterized by large water depths, i.e. by relative water depths of over 0.5. The dominant physical processes are: wind driven waves, whitecapping dissipation and non-linear quadruplet interactions.
\item {\bf the continental seas and the medium depths}, characterized by a relative water depth ranging from 0.05 to 0.5. In addition to the above processes, the bottom friction, the shoaling (wave growth due to a bottom rise) and the effects of refraction due to the bathymetry and/or to the currents are to be taken into account.
\item {\bf The coastal domain}, including shoals or near-shore areas (relative water depth lower than 0.05). For these shallow water areas, such physical processes as bottom friction, bathymetric breaking, non-linear triad interactions between waves should be included. Furthermore, it could be useful to take into account the effects related to unsteady sea level and currents due to the tide and/or to the weather-dependent surges.
\end{itemize}

Through a so-called finite element spatial discretization, one computational grid may include mesh cells among which the ratio of the largest sizes to the smallest ones may reach or even exceed 100. That is why TOMAWAC can be applied to a sea domain that is featured by highly variable relative water depths; in particular, the coastal areas can be finely represented.

The application domain of TOMAWAC does not include the harbour areas and, more generally, all those cases in which the effects of reflection on structures and/or diffraction may not be ignored.

A first version of a diffraction model is available in TOMAWAC and is able to represent some diffraction effects. The model presents still some limits. It is highly recommended to use phase-resolving models when a detailed simulation of diffraction effects is required (e.g. harbor agitation).

\subsubsection{Wave interactions with other physical factors}
Several factors are involved in the wave physics and interact to various extents with the waves changing their characteristics. The following main factors should be mentioned:
\begin{itemize}
\item bathymetry and sea bottom geometry (bottom friction, refraction, surf-breaking, non-linear effects of interactions with the bottom, sand rippling...)
\item atmospheric circulation (wind and pressure effects)
\item tide pattern (variation of currents and water heights),
\item three-dimensional oceanic circulation currents,
\item over/underelevations caused by exceptional weather events, resulting in sea levels variations up to several meters (storm, surges).
\end{itemize}
The fine modelling of the interactions between these various physical factors and the waves is generally rather complex and several research projects are currently focused on it. Within the application domain as defined in the previous paragraph, TOMAWAC models the following interactions:
\begin{itemize}

\item {\bf wave-bathymetry interaction}: the submarine relief data input into TOMAWAC are constant in time, but the sea level can change in time. In addition to the effects of the sea level variations in time, TOMAWAC allows to take into account refraction, shoaling, bottom friction and bathymetric breaking. TOMAWAC simulations can take into account some diffraction effects.
\item {\bf wave-atmosphere interaction}: this interaction is the driving phenomenon in the wave generation, takes part in energy dissipation processes (whitecapping, wave propagation against the wind…) and is involved in the energy transfer. To represent the unsteady behaviour of this interaction, TOMAWAC requires 10 m wind fields (specification of the couple of horizontal velocity components) with a time step matched to the weather conditions being modelled. These wind fields can be provided either by a meteorological model or from satellite measurements.
\item {\bf wave-current interaction}: the sea currents (as generated either by the tide or by oceanic circulations) may significantly affect the waves according to their intensity. They modify the refractive wave propagation direction, they reduce or increase the wave height according to their propagation direction in relation to the waves and may influence the wave periods if exhibiting a marked unsteady behaviour. In TOMAWAC, the current field is provided by the couple of horizontal components of its average (or depth-integrated) velocity at the nodes of the computational grid. TOMAWAC allows to model the frequency changes caused either by the Doppler effect or by the unsteady currents, as well as by a non-homogeneous current field.
\end{itemize}
\subsubsection{ The physical processes modelled in TOMAWAC}
Those interactions being taken into account by TOMAWAC have been reviewed and a number of physical events or processes have been mentioned in the previous paragraph. These processes modify the total wave energy as well as the directional spectrum distribution of that energy (i.e. the shape of the directional spectrum of energy). So far, the numerical modelling of these various processes, although some of them are now very well known, is not yet mature and keep on providing many investigation subjects. Considering the brief review of physical interactions given in the previous paragraph, the following physical processes are taken into account and digitally modelled in TOMAWAC:

{\bf—> Energy source/dissipation processes:}
\begin{itemize}
\item wind driven interactions with atmosphere. Those interactions imply the modelling of the wind energy input into the waves. It is the prevailing source term for the wave energy directional spectrum. The way that spectrum evolves primarily depends on wind velocity, direction, time of action and fetch (distance over which the wind is active). It must be pointed out that the energy which is dissipated when the wind attenuates the waves is not taken into account in TOMAWAC.
\item 	whitecapping dissipation or wave breaking, due to an excessive wave steepness during wave generation and propagation.
\item 	bottom friction-induced dissipation, mainly occurring in shallow water (bottom grain size distribution, ripples, percolation...)
\item 	dissipation through bathymetric breaking. As the waves come near the coast, they swell due to shoaling until they break when they become to steep.
\item dissipation through wave blocking due to strong opposing currents.
\end{itemize}
{\bf—> Non-linear energy transfer conservative processes:}
\begin{itemize}
\item 	non-linear resonant quadruplet interactions, which is the exchange process prevailing at great depths.
\item 		non-linear triad interactions, which become the prevailing process at small depths.
\end{itemize}
{\bf—> Wave propagation-related processes:}
\begin{itemize}
\item 	 wave propagation due to the wave group velocity and, in case, to the velocity of the medium in which it propagates (sea currents).
\item 	depth-induced refraction which, at small depths, modifies the directions of the wave-ray and then implies an energy transfer over the propagation directions.
\item 	shoaling: wave height variation process as the water depth decreases, due to the reduced wavelength and variation of energy propagation velocity.
\item 	current-induced refraction which also causes a deviation of the wave-ray and an energy transfer over the propagation directions.
\item 	interactions with unsteady currents, inducing frequency transfers (e.g. as regards tidal seas).
\item 	diffraction by a coastal structure (breakwater, pier, etc…) or a shoal, resulting in an energy transfer towards the shadow areas beyond the obstacles blocking the wave propagation. The current version of the diffraction model implemented in TOMAWAC is able to represent qualitatively some diffraction effects.
\end{itemize}

It should be remembered that, due to the hypothesis adopted in paragraph \ref{par31} about the TOMAWAC application domain, the reflection (partial or total) from a structure or a pronounced depth irregularity is not addressed by the model.

% %==================================
\section{Validation}
% %==================================
% As \estel is mainly used 
%
%==================================
\subsection{Evolution compared to the previous release}
%==================================
Beside some bugs corrected from the previous version, we can denote two main new phenomenas that are now modelized. 
\begin{itemize}
\item {\bf Strong dissipation through wave blocking due to strong opposing currents.} When water waves meet a strong adverse current, with a velocity that approaches the wave group velocity, waves are blocked. Two options can be now considered in TOMAWAC to take into account wave blocking effects. The first one considers an equilibrium range spectrum (in the presence of ambient flow) applied as an upper limit for the spectrum  \cite{hedges}. The second one add a dissipative term on the right-hand side of the action balance equation \cite{Westhuysen_2012}. This leads to one test case {\it called opposing\_current} that tests the two options.  
\item {\bf Dissipation due to vegetation.} When the ratio between vegetation height and water depth is important, vegetation can imply some dissipation. Some methods exist  to modelize this dissipation. The method we chose is based on the formulation proposed by Suzuki et al. \cite{suzuki}. This functionnality lead to 2 new test case called {\it dean } and {\it veget}. The first one is a realistic simulation with a fortran user file, the second one is less realistic but without any fortran user file. 
\end {itemize} 
%==================================
\subsection{Difference with the previous validation}
%==================================
In this validation we pay attention on two things. On one hand, we try to modify some old test case so that they could be run with the new version. On the other hand, in each test case we try to limit fortran user files. 

We can also remark tat due to the fact it is  the first time that this form of documentation is used, and the fact that previous documentation is quite old, differences in the form are quite important.   
%==================================
%\subsection{Potentialities tested}
%==================================


%==================================
%\subsection{Architectures}
%==================================

%==================================
%\subsection{Test case report}
%==================================
