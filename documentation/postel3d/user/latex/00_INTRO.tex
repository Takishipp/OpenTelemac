%==================================
%==================================
\section{Introduction}
%==================================
%==================================
%==================================
\subsection{A word of caution}
%==================================
This document contains information about the quality of a complex modelling tool. Its purpose is to assist the user in assessing the reliability and accuracy of computational results, and to provide guidelines with respect to the applicability and judicious employment of this tool. This document does not, however, provide mathematical proof of the correctness of results for a specific application. The reader is referred to the License Agreement for pertinent legal terms and conditions associated with the use of the software.

The contents of this validation document attest to the fact that computational modelling of complex physical systems requires great care and inherently involves a number of uncertain factors. In order to obtain useful and accurate results for a particular application, the use of high-quality modelling tools is necessary but not sufficient. Ultimately, the quality of the computational results that can be achieved will depend upon the adequacy of available data as well as a suitable choice of model and modelling parameters.
% 
\subsection{Document plan}
%==================================

%==================================
\subsection{Validation layout}
% %==================================

This validation is presented hereafter using a \textit{validation sheet form},
each sheet detailing the physical concepts involved, the physical and numerical parameters used and comparing both numerical and reference solutions.
Then, each sheet displays the following informations:
\begin{list}{-}{}
\item [-] \textbf{Purpose \& Problem description} : These first two parts give reader short details about the test case, the physical phenomena involved and specify how the numerical solution will be validated;
\item [-] \textbf{Reference} : This part gives the reference solution we are comparing to and explicits the analytical solution when available;
\item [-] \textbf{Physical parameters} : This part specifies the geometry,
details all the physical parameters used to describe both porous media (soil model in particularly) and
solute characteristics (dispersion/diffusion coefficients, soil $\equiv$ pollutant interactions...);
\item [-] \textbf{Geometry and Mesh} : This part describes the mesh used in the \sisyphe computation;
\item [-] \textbf{Initial and boundary conditions} : this part details both initial and boundary conditions used to simulate the case ;
\item [-] \textbf{Numerical parameters} : this part is used to specify the numerical parameters used
(adaptive time step, mass-lumping when necessary...);
\item [-] \textbf{Results} : we comment in this part the numerical results against the reference ones,
giving understanding keys and making assumptions when necessary.
\end{list}
%
\bigskip
%
\clearpage
%==================================
%==================================
\section{Presentation}
%==================================
%==================================
\subsection{General}
%==================================

%==================================
\subsection{Capabilities}
%==================================
% 
% %==================================
\section{Validation}
% %==================================
% As \estel is mainly used 
%
%==================================
\subsection{Evolution compared to the previous release}
%==================================

%==================================
\subsection{Difference with the previous validation}
%==================================

%==================================
\subsection{Potentialities tested}
%==================================


%==================================
\subsection{Architectures}
%==================================

\subsection{Test case report}
%==================================
