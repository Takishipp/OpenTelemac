% case name
\chapter{malpasset}
%
% - Purpose & Description:
%     These first two parts give reader short details about the test case,
%     the physical phenomena involved, the geometry and specify how the numerical solution will be validated
%
\section{Purpose}
%
This test illustrates that \telemac{2d} is able to simulate a real dam
break flow on an initially dry domain.
It also shows the propagation of the wave front and the evolution in
time of the water surface and velocities in the valley downstream.
%
\section{Description}
%
This case is the simulation of the propagation of the wave following
the break of the Malpasset dam (South-East of France).
Such accident really occurred in December 1959.
The model represents the reservoir upstream from the dam and the valley
and flood plain downstream.
The entire valley is approximately 18~km long and between 200~m (valley)
and 7~km wide (flood plain).
The complete study is described in details in [1].
The simulation is performed using the treatment of negative depths
introduced since \telemac{2d} 7.0.
The historical simulation using the method of characteristics
(named "charac") has been kept.
Nevertheless, the recommended advection scheme for velocities for such
applications is now the NERD scheme (14).
A simulation using a large mesh (named "large") is also performed.
%
% - Reference:
%     This part gives the reference solution we are comparing to and
%     explicits the analytical solution when available;
%
% bibliography can be here or at the end
%
%
\subsection{Reference}
%
[1] Hydrodynamics of Free Surface Flows modelling with the finite
element method. Jean-Michel Hervouet (Wiley, 2007) pp. 281-288.
%
% - Geometry and Mesh:
%     This part describes the mesh used in the computation
%
%
\subsection{Geometry and Mesh}
%
\subsubsection{Bathymetry}
%
Real topography % (see figures 3.8.1 and 3.8.2)
%
\subsubsection{Geometry}
%
Size of the model domain $\approx$ 17~km $\times$ 9~km
%
\subsubsection{Mesh}
%
The mesh is refined in the river valley (downstream from the dam)
and on the banks.\\
Regular mesh: 26,000 triangular elements / 13,541 nodes.
%(see figure 3.8.1).
Maximum size range is from 17 to 313~m\\
Large mesh: 104,000 triangular elements / 53,081 nodes.
%(see figure 3.8.2).
Maximum size range is from 8.5 to 156.5~m
%
% - Physical parameters:
%     This part specifies the physical parameters
%
%
\subsection{Physical parameters}
%
Constant viscosity equal to 1~m$^2$/s on horizontal directions\\
Coriolis: no\\
Wind: no
%
% Experimental results (if needed)
%\subsection{Experimental results}
%
% bibliography can be here or at the end
%\subsection{Reference}
%
% Section for computational options
%\section{Computational options}
%
% - Initial and boundary conditions:
%     This part details both initial and boundary conditions used to simulate the case
%
%
\subsection{Initial and Boundary Conditions}
%
\subsubsection{Initial conditions}
%
Full reservoir at initial time\\
No water in the downstream valley\\
No velocity
%
\subsubsection{Boundary conditions}
%
Channel banks: solid boundary without roughness (slip conditions)\\
Bottom: solid boundary with roughness.
Strickler formula with friction coefficient = 30~m$^{1/3}$/s\\
Solid boundary everywhere
%
\subsection{General parameters}
%
Time step: 4~s for regular mesh cases
(except the 1$^{\rm{st}}$ Order Kinetic scheme with 1~s and the primitive
equations with 0.5~s) and 1~s for large mesh case\\
Simulation duration: 4,000~s
%
% - Numerical parameters:
%     This part is used to specify the numerical parameters used
%     (adaptive time step, mass-lumping when necessary...)
%
%
\subsection{Numerical parameters}
%
Advection of velocities:
NERD scheme (14) for tidal flats with the treatment of negative
depths (regular + large meshes: "pos" and "large"),
but also just for regular mesh:
the new ERIA scheme ("ERIA", number 15),
the historical method of characteristics ("charac"),
the 1$^{\rm{st}}$ Order Kinetic scheme ("cin")
and the coupled primitive equations ("prim").
%
\subsection{Comments}
%
% - Results:
%     We comment in this part the numerical results against the reference ones,
%     giving understanding keys and making assumptions when necessary.
%
%
\section{Results}
%
Figure ??? illustrates the progression of the flood wave after the dam
break (simulation using the treatment of negative depths that smoothes
the results on tidal flats).
The propagation of the wave front is very fast.
The water depths increase rapidly in the valley downstream from the dam
location.
The wave spreads in the plain when arriving to the sea.
During the simulation, no negative water depths are observed.
%Figure ??? presents the water depth at time = 400~s obtained with 2
%planes (p2), 6 planes (p6) and the treatment of negative depths,
%2 planes using the method of characteristics (charac)
%and 2 planes using the fine mesh (p2\_large) calculations.
%The results are similar.
%However, the time profiles concerning the point located as indicated by
%squares in the upper panels of the figure, show that the "pos"
%planes simulation slightly underestimate the maximum water elevation.
%
\section{Conclusion}
%
\telemac{2d} is capable of simulating the propagation of a dam break
wave in a river valley initially dry.
%
% Here is an example of how to include the graph generated by validateTELEMAC.py
% They should be in test_case/img
%\begin{figure} [!h]
%\centering
%\includegraphics[scale=0.3]{../img/mygraph.png}
% \caption{mycaption}\label{mylabel}
%\end{figure}
%
% bibliography
%\section{Reference}
