% case name
\chapter{stratification}
%
% - Purpose & Description:
%     These first two parts give reader short details about the test case,
%     the physical phenomena involved, the geometry and specify how the numerical solution will be validated
%
\section{Purpose}
%
This test demonstrates the ability of \telemac{3d} to model stratified
flow with a special focus on the stability of the stratification.
This case also demonstrates the capacity of the $k$-$\epsilon$ model
to generate turbulent phenomenon.
%
\section{Description}
%
We consider a channel 2,000~m long and 100~m wide.
The bottom of this channel has a very mild slope.
The general water depth is 10~m and the velocity along the channel is
constant and equal to 1~m/s.
Salinity is prescribed as shown on figure 3.10.2.
%
% - Reference:
%     This part gives the reference solution we are comparing to and
%     explicits the analytical solution when available;
%
% bibliography can be here or at the end
%\subsection{Reference}
%
%
\subsection{Reference}
%

%
% - Geometry and Mesh:
%     This part describes the mesh used in the computation
%
%
\subsection{Geometry and Mesh}
%
\subsubsection{Bathymetry}
%
Mild slope bottom (0 at the entrance, -1.9~cm at the output)
%
\subsubsection{Geometry}
%
Channel length = 2,000~m\\
Channel width = 100~m (see figure 3.10.1)
%
\subsubsection{Mesh}
%
2,204 triangular elements\\
1,188 nodes\\
21 planes regularly spaced on the vertical
%
% - Physical parameters:
%     This part specifies the physical parameters
%
%
\subsection{Physical parameters}
%
Horizontal and vertical turbulence models: $k$-$\epsilon$ model\\
Bottom friction: Strickler coefficient = 70~m$^{1/3}$/s\\
Coriolis: no\\
Wind: no
%
% Experimental results (if needed)
%\subsection{Experimental results}
%
% bibliography can be here or at the end
%\subsection{Reference}
%
% Section for computational options
%\section{Computational options}
%
% - Initial and boundary conditions:
%     This part details both initial and boundary conditions used to simulate the case
%
%
\subsection{Initial and Boundary Conditions}
%
\subsubsection{Initial conditions}
%
Constant longitudinal velocity = 1~m/s\\
Constant water depth = 10~m\\
Constant tracer = 38 at the bottom below the plane number 18 and 28 at
the top above the plane number 18
%
\subsubsection{Boundary conditions}
%
Upstream: prescribed flow rate and tracer.\\
Downstream: prescribed water depth\\
Bottom: solid boundary with roughness\\
Lateral wall: no friction
%
\subsection{General parameters}
%
Time step: 2~s\\
Simulation duration 2,000~s
%
% - Numerical parameters:
%     This part is used to specify the numerical parameters used
%     (adaptive time step, mass-lumping when necessary...)
%
%
\subsection{Numerical parameters}
%
Advection of velocities, tracer and $k$-$\epsilon$: N-type MURD scheme
%
\subsection{Comments}
%
The tracer and velocity field are directly included in the
\telkey{CONDIM} subroutine.
%
% - Results:
%     We comment in this part the numerical results against the reference ones,
%     giving understanding keys and making assumptions when necessary.
%
%
\section{Results}
%
The vertical gradient of salinity remains stable as shown on figure
3.10.2.
On figure 3.10.3, we can observe the generation of the turbulence
phenomenon by the $k$-$\epsilon$ model.
This turbulence is created at the bottom and is developing on the
vertical column of water.
However, the turbulence is clearly blocked by the saline stratification.
%
\section{Conclusion}
%
\telemac{3d} is able to represent correctly stratified flow.
In addition, the $k$-$\epsilon$ model is able to simulate turbulence
generated by bottom friction.
%
% Here is an example of how to include the graph generated by validateTELEMAC.py
% They should be in test_case/img
%\begin{figure} [!h]
%\centering
%\includegraphics[scale=0.3]{../img/mygraph.png}
% \caption{mycaption}\label{mylabel}
%\end{figure}
%
% bibliography
%\section{Reference}
