% case name
\chapter{vent}
%
% - Purpose & Description:
%     These first two parts give reader short details about the test case,
%     the physical phenomena involved, the geometry and specify how the numerical solution will be validated
%
\section{Purpose}
%
This test demonstrates the availability of \telemac{3d} to represent
the currents induced by wind blowing at the surface of a closed channel.
More precisely, this case allows verifying that a linear decrease of
the mixing length model at the bottom and at the surface is able to
reproduce such circulation.
Moreover, this test allows verifying the proper implementation of the
sources terms and external forces as the wind.
%
\section{Description}
%
At the initial state, the channel is submitted to a constant 10~m/s
wind.
The wind generates a slope of the free surface and a vertical
two-dimensional circulation.
Tsanis [1] has made an inventory of existing laboratory or in-situ
measurements and has plotted these values on a non-dimensional graph.
He deduced a characteristic vertical velocity profile, which will be
compared to the numerical results of this test case.
The turbulent viscosity has the following expression:
\begin{equation}
\nu_{\rm{t}} = l^2 \sqrt{\frac{1}{2}\left[ \frac{\partial u_i}{\partial x_j}
 + \frac{\partial u_j}{\partial x_i} \right]^2}
\end{equation}

$l$ = mixing length\\
$K$ = 0.40 Karman constant\\

The channel is a rectangle of 500~m by 100~m with horizontal bed at
depth $z$ = -10~m.\\
With:
\begin{equation}
\vec w = 10~\rm{m/s~Velocity~of~wind}
\end{equation}
\begin{equation}
K_{\rm{F}} = (-0.12 + 0.137 \| \vec w \| )/1000
\end{equation}
%
% - Reference:
%     This part gives the reference solution we are comparing to and
%     explicits the analytical solution when available;
%
% bibliography can be here or at the end
%\subsection{Reference}
%
%
\subsection{Reference}
%
[1] TSANIS I., Simulation of wind-induced water currents,
Journal of Hydraulic Engineering, Vol.115, n 8, 1989, pp 1113-1134.\\

[2] WU J., Prediction of near-surface drift currents from wind velocity,
Journal of Hydraulic Division, 99, 1973, pp 1291-1302.
%
% - Geometry and Mesh:
%     This part describes the mesh used in the computation
%
%
\subsection{Geometry and Mesh}
%
\subsubsection{Bathymetry}
%
Constant equal to –10~m
%
\subsubsection{Geometry}
%
Channel length = 500~m\\
Channel width = 100~m
%
\subsubsection{Mesh}
%
543 triangular elements\\
315 nodes\\
15 planes irregularly spaced on the vertical (see figure 3.4.1
generated with a vertical expansion factor equal to 20)
%
% - Physical parameters:
%     This part specifies the physical parameters
%
%
\subsection{Physical parameters}
%
Vertical turbulence model: Tsanis mixing length turbulence model\\
Horizontal viscosity for velocity: 0.1 m$^2$/s\\
Coriolis: no\\
Wind: 10 m/s in $x$ direction
%
% Experimental results (if needed)
%\subsection{Experimental results}
%
% bibliography can be here or at the end
%\subsection{Reference}
%
% Section for computational options
%\section{Computational options}
%
% - Initial and boundary conditions:
%     This part details both initial and boundary conditions used to simulate the case
%
%
\subsection{Initial and Boundary Conditions}
%
\subsubsection{Initial conditions}
%
No velocity, initial water level equal to zero
%
\subsubsection{Boundary conditions}
%
Channel banks: solid slip boundary\\
Bottom: solid boundaries with friction stress due to mixing length\\
Surface: shear stress wind
%
\subsection{General parameters}
%
Time step: 10~s\\
Simulation duration: 20,000~s (5~h 33~min 20~s)
%
% - Numerical parameters:
%     This part is used to specify the numerical parameters used
%     (adaptive time step, mass-lumping when necessary...)
%
%
\subsection{Numerical parameters}
%
Hydrostatic computation\\
Advection for velocities: PSI-type MURD scheme\\
Diagonal preconditioning for propagation
%
\subsection{Comments}
%
% - Results:
%     We comment in this part the numerical results against the reference ones,
%     giving understanding keys and making assumptions when necessary.
%
%
\section{Results}
%
The mass balance is the following:

\begin{lstlisting}[language=TelFortran]
   MASSE INITIALE (DEBUT DE CE CALCUL) : 500000.0
   MASSE FINALE                      : 500000.0
   MASSE SORTIE DU DOMAINE (OU SOURCE) : 0.000000
   MASSE PERDUE                       : -0.5280007E-03
\end{lstlisting}

Thus, the mass balance is consistent with the accuracy asked
(from 10$^{-8}$ to 10$^{-6}$ depending on the system to solve,
for the diffusion of velocities, propagation or vertical velocity).
Figure 3.4.2 shows the vertical circulation induced by the wind.
The velocity at the surface is 18 cm/s and the return current reaches
a maximum value of 5.7 cm/s.
However, it must be pointed out that the velocity at the surface depends
on the refinement near the surface because the velocity gradient is
very high in this area.
A distance between the two first vertical points of 0.50~m instead of
0.10~m induced a velocity at the surface of 12~cm/s, but the velocity
field below 1~m under the surface was only slightly modified.
J. Wu [2] proposed for the velocity us at the surface the expression:
\begin{equation}
u_s = 0.55 (\tau/\rho_{air})^{1/2}
\end{equation}.
This gives $u_s$ = 19~cm/s.
Then, the velocity computed with a mesh of 0.10~m is very close to
this theoretical value.
Figure 3.4.2 shows the non-dimensional plot of the vertical velocity
profile.
The numerical results fit the measurements reasonably well.
The upper part of the profile, where the velocities have the same
orientation as the wind, is close to measured profiles, but the lower
part is smoothed.
This could mean that the level of turbulence is stronger in nature.
The slope of the free surface presented in figure 3.4.3 is equal to
1.576 10$^{-6}$.
The computation of the slope, assuming that the flow is homogeneous on
the vertical, gives a slope equals to 1.66 10$^{-6}$.
This value is in agreement with the value given by \telemac{3d}.
%
\section{Conclusion}
%
The velocity field produced by \telemac{3d} using the standard mixing
length is correct.
Near the surface, the quality of the results depends on the vertical
resolution near the surface.
The second points of the vertical mesh should be 10~cm below the surface
for a good result.
In the deeper part, the profile is a little bit more smoothed.
Taking into account the effect of the wind directly into the turbulence
model may ameliorate this result.
%
% Here is an example of how to include the graph generated by validateTELEMAC.py
% They should be in test_case/img
%\begin{figure} [!h]
%\centering
%\includegraphics[scale=0.3]{../img/mygraph.png}
% \caption{mycaption}\label{mylabel}
%\end{figure}
%
% bibliography
%\section{Reference}
