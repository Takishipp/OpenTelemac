\subsection{depot}
%

% - Purpose & Problem description: 
%     These first two parts give reader short details about the test case, 
%     the physical phenomena involved and specify how the numerical solution will be validated
%    
\subsubsection{Purpose}
%

%
\subsubsection{Description of the problem}
%

% - Reference: 
%     This part gives the reference solution we are comparing to and 
%     explicits the analytical solution when available;
%    
%
\subsubsection{Reference}
%

% - Physical parameters: 
%     This part specifies the geometry, details all the physical parameters 
%     used to describe both porous media (soil model in particularly) and
%     solute characteristics (dispersion/diffusion coefficients, soil <=> pollutant interactions...)
%
%
\subsubsection{Physical parameters}
%

% - Geometry and Mesh: 
%     This part describes the mesh used in the computation
%
%
\subsubsection{Geometry and Mesh}
%

% - Initial and boundary conditions: 
%     This part details both initial and boundary conditions used to simulate the case 
%
%
\subsubsection{Initial and Boundary Conditions}
%

% - Numerical parameters: 
%     This part is used to specify the numerical parameters used
%     (adaptive time step, mass-lumping when necessary...)
%
%
\subsubsection{Numerical parameters}
%

% - Results: 
%     We comment in this part the numerical results against the reference ones,
%     giving understanding keys and making assumptions when necessary.
%
%
\subsubsection{Results}
%

% Here is an example of how to include the graph generated by validateTELEMAC.py
% They should be in test_case/img
%\begin{figure} [!h]
%\centering
%\includegraphics[scale=0.3]{../img/mygraph.png}
% \caption{mycaption}\label{mylabel}
%\end{figure}


