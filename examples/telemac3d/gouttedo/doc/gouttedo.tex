% case name
\chapter{gouttedo}
%
% - Purpose & Description:
%     These first two parts give reader short details about the test case,
%     the physical phenomena involved, the geometry and specify how the numerical solution will be validated
%
\section{Purpose}
%
This test demonstrates that the \telemac{3d} solution is not polarised
because it can simulate the circular spreading of a wave in a square
domain.
It also shows that the no-flow condition is satisfied on solid
boundaries and that the solution remains symmetric after reflection
of the circular wave on the boundaries.
%
\section{Description}
%
% - Reference:
%     This part gives the reference solution we are comparing to and
%     explicits the analytical solution when available;
%
% bibliography can be here or at the end
%\subsection{Reference}
%
The fluid is initially at rest with a Gaussian free surface in the
centre of a square domain (see figure 3.3.2).
The evolution of the surface and the reflection of the wave by the
solid boundaries are then calculated during 4 seconds.
%
\subsection{Reference}
%

%
% - Geometry and Mesh:
%     This part describes the mesh used in the computation
%
%
\subsection{Geometry and Mesh}
%
\subsubsection{Bathymetry}
%
Flat bottom
%
\subsubsection{Geometry}
%
Square length = 20.1~m
%
\subsubsection{Mesh}
%
8,978 triangular elements\\
4,624 nodes\\
Vertical Mesh: 3 planes regularly spaced on the vertical
(see figure 3.3.1)
%
% - Physical parameters:
%     This part specifies the physical parameters
%
%
\subsection{Physical parameters}
%
Turbulence: constant viscosity in both directions
(molecular viscosity)\\
Bottom friction: Chézy law with coefficient equal to 60~m$^{1/2}$/s\\
Coriolis: no\\
Wind: no
%
% Experimental results (if needed)
%\subsection{Experimental results}
%
% bibliography can be here or at the end
%\subsection{Reference}
%
% Section for computational options
%\section{Computational options}
%
% - Initial and boundary conditions:
%     This part details both initial and boundary conditions used to simulate the case
%
%
\subsection{Initial and Boundary Conditions}
%
\subsubsection{Initial conditions}
%
Water depth at boundary: 2.4~m\\
Water depth at the centre: 4.8~m\\
No velocity
%
\subsubsection{Boundary conditions}
%
Solid wall with slip condition
%
\subsection{General parameters}
%
Time step: 0.04~s\\
Simulation duration: 4~s
%
% - Numerical parameters:
%     This part is used to specify the numerical parameters used
%     (adaptive time step, mass-lumping when necessary...)
%
%
\subsection{Numerical parameters}
%
Non-hydrostatic computation\\
Advection for velocities: PSI-type MURD scheme
%
\subsection{Comments}
%
The initial free surface elevation is prescribed in the \telkey{CONDIM}
subroutine.
%
% - Results:
%     We comment in this part the numerical results against the reference ones,
%     giving understanding keys and making assumptions when necessary.
%
%
\section{Results}
%
The wave spreads circularly around the initial water surface peak
elevation (Figure 3.3.2 to Figure 3.3.4).
When it reaches the boundaries, reflection occurs.
The reflected wave is also axi-symmetric.
The final volume in the domain is equal to the initial volume.
%
\section{Conclusion}
%
Even though the mesh is polarised (along the $x$ and $y$ directions and
the main diagonal), the solution is not.
Solid boundaries are treated properly: no bias occurs in the reflected
wave.
Water mass is conserved.
%
% Here is an example of how to include the graph generated by validateTELEMAC.py
% They should be in test_case/img
%\begin{figure} [!h]
%\centering
%\includegraphics[scale=0.3]{../img/mygraph.png}
% \caption{mycaption}\label{mylabel}
%\end{figure}
%
% bibliography
%\section{Reference}
