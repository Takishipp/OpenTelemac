% case name
\chapter{bump}
%
% - Purpose & Description:
%     These first two parts give reader short details about the test case,
%     the physical phenomena involved, the geometry and specify how the numerical solution will be validated
%
\section{Purpose}
%
This test calculates the fluvial regime in a horizontal straight channel
including a topographical singularity.
This problem has a bidimensionnal analytical solution in 2D.
Therefore, it allows testing the accuracy on the computation of the
free-surface, with respect to the bottom gradient.
%
\section{Description}
%
For a given discharge per unit length $Q$, imposed at the upstream
boundary and a water depth $H$ imposed at the downstream boundary,
the water line fulfils the Bernoulli equations:
\begin{equation}
\frac{Q^2}{2gh^3}+ (h + Z_{\rm{f}}) = H_0,
\end{equation}
with:\\
\begin{tabular}{rl}
$H_0$ & specific energy\\
$h$   & water depth\\
$Z_f$ & bottom elevation
\end{tabular}\\

The profile of the singularity is defined by:
\begin{equation}
Z_{\rm{f}} = 3 \sin^2 \left( \frac{x - \frac{1}{2} (L - L_0)}{L_0} \right),
\end{equation}
with:\\
\begin{tabular}{rl}
$L$  & length of the channel\\
$L0$ & width of the bump\\
$x$  & abscissa of the point
\end{tabular}
%
% - Reference:
%     This part gives the reference solution we are comparing to and
%     explicits the analytical solution when available;
%
% bibliography can be here or at the end
%\subsection{Reference}
%
%
\subsection{Reference}
%
Hydrodynamics of Free Surface Flows modelling with the finite element
method. Jean-Michel Hervouet (Wiley, 2007) pp 128-129.
%
% - Geometry and Mesh:
%     This part describes the mesh used in the computation
%
%
\subsection{Geometry and Mesh}
%
\subsubsection{Bathymetry}
%
Flat straight channel with a bump in the centre of the channel
(figure 3.6.1)\\
Bottom of the flat part of the channel at -0.2~m\\
Height of the bump is 20~cm ($x$ = 10~cm)
%
\subsubsection{Geometry}
%
Channel length = 20.96~m\\
Channel width = 2~m (figure 3.6.1)
%
\subsubsection{Mesh}
%
2,620 triangular elements\\
1,452 nodes\\
The mesh is made up of squares whose sides measure 16~cm cut into
triangles\\
5 layers regularly spaced on the vertical (see figure 3.6.1)
%
% - Physical parameters:
%     This part specifies the physical parameters
%
%
\subsection{Physical parameters}
%
Constant horizontal viscosity = 10$^{-6}$ m$^2$/s\\
Vertical turbulence model: mixing length model\\
Coriolis: no\\
Wind: no
%
% Experimental results (if needed)
%\subsection{Experimental results}
%
% bibliography can be here or at the end
%\subsection{Reference}
%
% Section for computational options
%\section{Computational options}
%
% - Initial and boundary conditions:
%     This part details both initial and boundary conditions used to simulate the case
%
%
\subsection{Initial and Boundary Conditions}
%
\subsubsection{Initial conditions}
%
No velocity\\
Level of free surface at initial state: 40~cm
%
\subsubsection{Boundary conditions}
%
Channel banks: solid boundary without roughness\\
Bottom: solid boundary with roughness (Strickler = 50~m$^{1/3}$/s)\\
Upstream imposed discharge 2~m$^3$/s\\
Downstream imposed water level 0.4~m
%
\subsection{General parameters}
%
Time step 0.01~s\\
Simulation duration: 50~s
%
% - Numerical parameters:
%     This part is used to specify the numerical parameters used
%     (adaptive time step, mass-lumping when necessary...)
%
%
\subsection{Numerical parameters}
%
Non-hydrostatic simulation\\
Advection of velocities: characteristics
%
\subsection{Comments}
%
% - Results:
%     We comment in this part the numerical results against the reference ones,
%     giving understanding keys and making assumptions when necessary.
%
%
\section{Results}
%
Qualitatively the velocity field is regular at the surface and
vertically in the critical flow area.
It follows well the shape of the bump (Figure 3.6.2).
Generally, the results are in good agreement with the analytical
solution (described in details in [1]).
The position of the hydraulic jump is correctly computed.
However, a shift (difference of the free surface at the entrance) is
observed on the free surface, which seems to be due to the \telemac{2d}
calculus of the free surface used by \telemac{3d}.
%
\section{Conclusion}
%
This flow is well reproduced by \telemac{3d}.
%
% Here is an example of how to include the graph generated by validateTELEMAC.py
% They should be in test_case/img
%\begin{figure} [!h]
%\centering
%\includegraphics[scale=0.3]{../img/mygraph.png}
% \caption{mycaption}\label{mylabel}
%\end{figure}
%
% bibliography
%\section{Reference}
