\section{Suspended load}\label{ch:suspension}
\subsection{Suspended load transport equation}

\subsubsection{Passive scalar hypothesis}
For non-equilibrium flow conditions, the bedload and the suspended load
are dealt separately. The interface between the bedload and suspended
load is located at $z=Z_{ref}$:
\begin{itemize} 
\item In the thin high-concentrated bedload layer ($z <Z_{ref}$) inter-particle interactions and flow-turbulent interactions strongly modify the flow structure. Equilibrium conditions are however a
reliable assumption to relate the bed-load to the current induced bed shear
stress.
\item In the upper part of the flow ($z>Z_{ref}$), for dilute suspension clear flow concepts still apply, and the sediment grains can be regarded as a passive scalar which follows the mean and turbulent flow
velocity, with an additional settling velocity term.
\end{itemize}

\subsubsection{Settling velocity}
The settling velocity $w_{s}$ is an important parameter for the suspension.
It can be either specified or calculated by the model as a
function of grain diameter. The van Rijn formula~\cite{vanRijn87, vanRijn93} which is valid for non-cohesive spherical particles and dilute
suspensions, has been implemented in \sisyphe as follows:

\begin{equation*}
w_{s} = \left\{\begin{array}{ll}
\displaystyle
\frac{(s-1)g d_{50}^2}{18\nu}, & \quad \text{if } d_{50} \leq 10^{-4} \\
\displaystyle
\frac{10\nu}{d_{50}} \left(\sqrt{1+0.01\frac{(s-1)gd_{50}^3}{18\nu^2}}-1\right), & \quad \text{if } 10^{-4} \leq d_{50} \leq 10^{-3}\\ 
\displaystyle
1.1 \sqrt{(s-1)gd_{50}}, & \quad \text{otherwise} 
\end{array}
\right.
\end{equation*}
with $s=\rho_{s}/\rho_0$ is the relative density and $g$ is gravity.%

Outside the bedload layer sediment particles are assumed to follow the mean
flow velocity, $\mathbf{u} = (u,v,w)^T$, with an additional vertical
settling velocity $w_s$. 

\subsubsection{Three-dimensional sediment transport equation}
The suspended sediment load can be calculated by solving the full three-dimensional (3D) advection-diffusion equation for the suspended sediment concentration distribution, expressed as:

\begin{equation}\label{eq:ADE3D}
\frac{\partial c}{\partial t} + \frac{\partial(u c)}{\partial x} + \frac{\partial(v c)}{\partial y} + \frac{\partial(w c)}{\partial z} - \frac{\partial(w_s c)}{\partial z} = \frac{\partial}{\partial x}\left(\varepsilon_s\frac{\partial c}{\partial x}\right) + 
                                           \frac{\partial}{\partial y}\left(\varepsilon_s\frac{\partial c}{\partial y}\right) +
                                           \frac{\partial}{\partial z}\left(\varepsilon_s\frac{\partial c}{\partial z}\right) 
\end{equation}
with $c=c(\mathbf x,t)$ the volumetric suspended sediment concentration, $w_s > 0$ the vertical-settling sediment velocity, $\varepsilon_s$ the turbulent diffusivity of sediment, often related to the eddy viscosity $\nu_t$ by $\varepsilon_s=\nu_t/\sigma_c$, with $\sigma_c$ being the turbulent Schmidt number (between $0.5$ and $1.0$). The advection-diffusion equation (\ref{eq:ADE3D}) is completed with initial conditions specifying $C=C(\mathbf x,0)$ and boundary conditions as follows. 

At the inlet boundary, a local equilibrium concentration profile can be specified. This profile can be derived from equation (\ref{eq:ADE3D}) by assuming uniform and steady flow conditions. If a parabolic distribution of turbulent eddy diffusivity coefficient is adopted, then the vertical distribution of suspended sediment concentration is the classical Rouse profile.
At the outlet, the normal gradients of the concentration are set equal to zero. A similar boundary condition can be specified at the sidewalls of the model. 

At the free surface $z=Z_s$, the net vertical sediment flux is set to zero, thus:
\begin{equation*}
\left(\varepsilon_s\frac{\partial c}{\partial z} + w_s c\right)_{z=Z_s} = 0.
\end{equation*}

At the interface between the bed-load and the suspended load $z=Z_{ref}$, a Neumann type boundary condition is specified, in which the total vertical flux equals the net sediment transport:

\begin{equation*}
\left(\varepsilon_s\frac{\partial c}{\partial z} + w_sc\right)_{z=Z_{ref}} = D-E.
\end{equation*}

The deposition $D$ and entrainment $E$ fluxes can be expressed as $D=w_s c|_{z=z_b}$ and $E=w_s c_{ref}$, respectively. Further details are given next. The 3D advection-diffusion equation (\ref{eq:ADE3D}), with suitable initial and boundary conditions are solved with {\scshape Sedi-3d}, the set of subroutines of \telddd for three-dimensional sediment transport modelling. We refer the reader to the corresponding documentation~\cite{sedi3d}.   

\subsubsection{Two-dimensional sediment transport equation}
The two-dimensional (2D) sediment transport equation for the depth-averaged suspended-load concentration $C=C(x,y,t)$ is
obtained by integrating the 3D sediment transport equation (\ref{eq:ADE3D}) over the suspended-load zone. By applying the Leibniz integral rule, adopting suitable boundary conditions and assuming that the bedload zone is very thin, the depth-integrated suspended-load transport equation is obtained:
\begin{equation}\label{eq:ADE2D}
\frac{\partial hC}{\partial t} + \frac{\partial (hUC)}{\partial x} +
\frac{\partial (hVC)}{\partial y} = 
\frac{\partial }{\partial x} \left(h\epsilon_s\frac{\partial C}{\partial x}\right) + 
\frac{\partial }{\partial y} \left(h\epsilon_s\frac{\partial C}{\partial y}\right) + E - D,
\end{equation}
with $h=Z_s-Z_f \approx Z_s-Z_{zref}$ the water depth, assuming that the bedload layer thickness is very thin, and $(U,V)^T$ are the depth-averaged velocity components in the $x-$ and $y-$directions, respectively. 

If the effects of the heterogeneous vertical distribution of suspended sediment are taken into account, the following 2D depth-averaged horizontal suspended sediment transport model in its non-conservative form is obtained:
\begin{equation}\label{eq:ADE2DNC}
\frac{\partial C}{\partial t} + \underbrace{
U \frac{\partial C}{\partial x} + 
V \frac{\partial C}{\partial y}}_{\text{advection}} = 
\underbrace{\frac{1}{h} \left(\frac{\partial }{\partial x} \left( h\epsilon_s \frac{\partial C}{\partial x} \right) + 
\frac{\partial }{\partial y} \left(h\epsilon_s \frac{\partial C}{\partial y} \right) \right)}_{\text{diffusion}} +
\frac{E-D}{h}, 
\end{equation}
where $U$ and $V$ are the depth-averaged convective flow velocities in the $x-$ and $y-$ directions, respectively.

\begin{bclogo}[couleur = blue!10, arrondi = 0.10, logo = \bcattention]{\textsf{Attention}}
Different schemes are available for solving the non-linear advection terms
depending on the choice of the parameter the classical characteristics
schemes. To the diffusive schemes SUPG and PSI, it is recommended to use conservative schemes. 
\end{bclogo}

\subsubsection{Treatment of the diffusion terms}

According to the choice of the parameter \texttt{OPDTRA} of the keyword \texttt{OPTION FOR THE DIFFUSION OF TRACER}, the diffusion terms in (\ref{eq:ADE2DNC}) can be simplified as follows:

\begin{itemize}
\item \texttt{OPDTRA = 1}: the diffusion term is solved in the form $\grad\cdot(\varepsilon_s\grad T)$
\item \texttt{OPDTRA = 2}: the diffusion term is solved in the form $\frac{1}{h}\grad\cdot(h\varepsilon_s\grad T)$
\end{itemize}

The value of the dispersion coefficient $\epsilon_s$ depends on the
choice of the parameter \texttt{OPTDIF} of the keyword \texttt{OPTION FOR THE DISPERSION}:

\begin{itemize}
\item \texttt{OPTDIF = 1}: the values of the constant longitudinal and transversal dispersivity coefficients ($T1$ and $T2$ respectively) are provided with the keywords \texttt{DISPERSION ALONG THE FLOW} and \texttt{DISPERSION ACROSS THE FLOW} 
\item \texttt{OPTDIF = 2}: the values of the longitudinal and transversal dispersivity coefficients are computed with the Elder model $T1=\alpha_l u^* h$ and $T2=\_alpha_t u^* h$, where the coefficients $\alpha_l$ and $\alpha_t$ can be provided with the keywords \texttt{DISPERSION ALONG THE FLOW} and \texttt{DISPERSION ACROSS THE FLOW}. 
%the Elder model is used to calculate both the longitudinal $\epsilon^l$ and transversal $\epsilon^t$ dispersion components as a function of the friction velocity as $\epsilon^l_s = \alpha_l u_*h$ and $\epsilon^t_s = \alpha_t u_*h$, with $\alpha_l=6$ and $\alpha_t=0.6$
\item \texttt{OPTDIF = 3}: the values of the dispersion coefficients are provide by \teldd
\end{itemize}

The diffusion term can also be set to zero with keyword \texttt{DIFFUSION = NO} (\texttt{YES} is the option by default).


%\medskip
%\begin{bclogo}[couleur=blue!10,arrondi=0.1, logo=\bcinfo]{Keywords}
%\begin{itemize}
%\item {\sffamily SUSPENSION} (= NO by default)
%\item {\sffamily DIFFUSION} (= YES, by default)
%\item {\sffamily OPTION FOR THE DIFFUSION OF TRACER}
%(OPDTRA=1,the default option corresponds to the simplified expression of the
%diffusion term)
%\item {\sffamily OPTION FOR THE DISPERSION} (=1, default option)
%\item {\sffamily DISPERSION ALONG THE FLOW} ($\epsilon_s = 10^{-2}$ m$^{2}$s$^{-1}$, default value)
%\item {\sffamily DISPERSION ACROSS THE FLOW} ($\epsilon_s = 10^{-2}$ m$^{2}$s$^{-1}$, default value)
%\item {\sffamily TYPE OF ADVECTION} (RESOL=1 , method of
%characteristics by default)
%\end{itemize}
%\end{bclogo}

\subsubsection{Treatment of the advection terms}
The convective velocity can be corrected from the depth-averaged
mean velocity in order to account for the fact that most sediment is
transported near the bed.

\begin{bclogo}[couleur = blue!10, arrondi = 0.10, logo = \bcattention]{\textsf{Tidal flats}}
In presence of tidal flats, it is recommended to use conservative
scheme based on the calculation of flux per segments. The scheme \texttt{14} should
be used only if the convection velocity differs from the depth-averaged velocity
and no longer satisfies the shallow-water continuity equation, that is when the keyword \texttt{CORRECTION ON CONVECTION VELOCITY = YES}.
\end{bclogo}

\subsection{Bed evolution}
\subsubsection{Bed evolution due to the suspension}
By considering only suspended load sediment transport, the bed evolution is function of the net vertical sediment flux at the interface between the bedload and the suspended
load given by: 
\begin{equation}\label{eq:ExnerSusp}
(1-n)\frac{\partial Z_f}{\partial t} + (E-D)_{z=Z_{ref}} = 0  
\end{equation}
with $Z_f$ the bed elevation, $Z_{ref}$ the interface between the
bedload and suspended load, and $n$ the bed porosity. Once the flow variables are determined by solving the hydrodynamics and suspended sediment transport equations, changes of bed level are computed from Equation (\ref{eq:ExnerSusp}) for the cell coincident to the bed, calculating at each time step the net sediment flux $E-D$ at the bedload-suspended load
interface ($z = Z_{ref}$), as explained later. Further details on the derivation of the mass balance equation (\ref{eq:ExnerSusp}) and coupling strategies can be found in~\cite{Wu}.

\begin{bclogo}[couleur = blue!10, arrondi = 0.10, logo = \bcattention]{\textsf{Attention}}
In Equation (\ref{eq:ExnerSusp}), the net sediment flux $D-E$ is expressed in ms$^{-1}$. 
For consistency with bedload units, expressed in m$^{2}$s$^{-1}$), concentration is dimensionless. 
However, the user can choose concentration by mass per unit volume of the mixture (kg m$^{-3}$) by using the keyword \texttt{MASS CONCENTRATION}. The relation between concentration by volume and concentration by mass is $C_m$ (kg m$^{-3}$) = $\rho_s C$.
\end{bclogo}

\subsection{Equilibrium concentrations}

\subsubsection{Erosion and deposition rates}
For non-cohesive sediments, the net sediment flux $E-D$ is
determined based on the concept of equilibrium concentration, see~\cite{CelikRodi}:
\begin{equation}\label{eq:CelikRodi}
\left(E-D \right)_{Z_{ref}} = W_s \left(C_{eq} - C_{Z_{ref}}\right) 
\end{equation}
where $C_{eq}$ is the equilibrium near-bed concentration and $C_{zref}$ is
the near-bed concentration, calculated at the interface between the bed-load
and the suspended load, $z=Z_{ref}$. The keyword \texttt{REFERENCE CONCENTRATION FORMULA} access to four differents semi-empirical formulas: 
\begin{itemize}
\item \texttt{ICQ= 1}: Zyserman and Fredsoe formula
\begin{equation}\label{eq:ZysermanFredsoe}
C_{eq} =\frac{0.331(\theta'-\theta_c)^{1.75}}{1+0.72(\theta'-\theta_c)^{1.75}}, 
\end{equation}
where $\theta_c$ is the critical Shields parameter and $\theta'= \mu\theta$ ,the non-dimensional skin friction which is related to the Shields parameter, by use of the ripple correction factor.

\item \texttt{ICQ= 2}: Bijker (1992) formula
The equilibrium concentration corresponds to the volume concentration at the
top of the bed-load layer. It can related to the bed load transport rate:
\begin{equation}\label{eq:Bijker}
C_{eq} = \frac{Q_s}{b Z_{ref} u_*}, 
\end{equation}
with an empirical factor $b = 6.34$.

\begin{bclogo}[couleur = blue!10, arrondi = 0.10, logo = \bcattention]{\textsf{Attention}}
The near bed concentration computed with the Bijker formula is related to the bedload. Therefore, this option cannot be used without bedload transport (\texttt{BED-LOAD = YES}). Furthermore, both bedload and suspended load transport must be calculated at each time step, with the keyword \texttt{PERIOD OF COUPLING} set equal to 1.
\end{bclogo}

\item \texttt{ICQ= 3}: van Rijn formula

\item \texttt{ICQ= 4}: Soulsby-van Rijn formula
\end{itemize}

\subsubsection{Reference level}
The reference elevation $Z_{ref}$ corresponds to the interface
between bedload and suspended load. For flat beds, the bed-load layer
thickness is proportional to the grain size, whereas when the bed is
rippled, the bedload layer thickness scales with the equilibrium bed
roughness ($k_r$). The definition of the reference elevation needs also to be consistent with
the choice of the equilibrium near-bed concentration formula. 

\begin{itemize}
\item \texttt{ICQ = 1}: According to Zyserman and Fredsoe, the reference
elevation should be set to $Z_{ref} = 2d_{50}$. In \sisyphe we take $Z_{ref}=k'_s$, 
where $k'_s$ is the skin bed roughness for flat bed ($k_s' \approx d_{50}$) 
, the default value of proportionality factor is \texttt{KSPRATIO =3}). 
\item \texttt{ICQ = 2}: According to Bijker, $Z_{ref} = k_r$.
\end{itemize}

\subsubsection{Vertical concentration profile}
We assume here a Rouse profile for the vertical concentration distribution,
which is theoretically valid in uniform steady flow conditions:
\begin{equation}\label{eq:Rouseprofile}
C(z)=C_{Z_{ref}}\left(\frac{z-h}{z}\frac{a}{a-h}\right)^R, 
\end{equation}
where $R$ is the Rouse number defined by
\begin{equation}\label{eq:R}
R=\frac{w_s}{\kappa u_*}, 
\end{equation}
with $\kappa$ the von Karman constant ($\kappa = 0.4$), $u_*$ the
friction velocity corresponding to the total bed shear stress (see \S~\ref{xxx}), and $a$ the reference elevation above the bed elevation.

\subsubsection{Ratio between the reference and depth-averaged concentration}
By depth-integration of the Rouse profile (\ref{eq:Rouseprofile}), the following relation
can be established between the mean (depth-averaged) concentration and the
reference concentration:
\begin{equation*}
C_{Zref} = F C,
\end{equation*}
where:
\begin{equation}\label{eq:Rouseprofile}
F^{-1} = \left(\frac{Z_{ref}}{h}\right)^R\int_{Zref/h}^1\left(\frac{1-u}{u}\right)^R du. 
\end{equation}
In \sisyphe, we use the following approximate expression for $F$:
\begin{align*}
F^{-1} &= \frac{1}{\left(1-Z\right)} B^R\left( 1-B^{(1-R)} \right), \quad \text{for } R \neq 1\\
F^{-1} &= -B \log B, \quad \text{for } R = 1,
\end{align*}
with $B = Z_{ref}/h$.

\subsection{Convection velocity}
A straightforward treatment of the advection terms would imply the
definition of an advection velocity and replacement of the depth-averaged
velocity $U$ in Eq. (\ref{eq:ADE2DNC}) by:
\begin{equation*}
U_{conv} = \overline{UC}/C.
\end{equation*}
A correction factor is introduced in \sisyphe, defined by:
\begin{equation*}
F_{conv} =\frac{U_{conv}}{U} 
\end{equation*}
The convection velocity should be smaller than the mean flow velocity ($F_{conv} \leq 1$) since sediment concentrations are mostly transported in the lower part of the water column where velocities are smaller. We further
assume an exponential profile concentration profile which is a reasonable
approximation of the Rouse profile, and a logarithmic velocity profile, in
order to establish the following analytical expression for $F_{conv}$:
\begin{equation*}
F_{conv} =-\frac{I_2 - \ln\left(\frac{B}{30}\right) I_1}{I_1 \ln\left( 
\frac{eB}{30}\right)}, 
\end{equation*}
with $B=k_s/h = Z_{ref}/h$ and 
\begin{equation*}
I_1 =\int_B^1\left(\frac{(1-u)}{u}\right)^R du,\quad I_2 = \int_B^1 \ln u \left(\frac{(1-u)}{u} \right)^R du.  
\end{equation*}
For further details, see~\cite{Huybrechts}.

\medskip
\begin{bclogo}[couleur=blue!10,arrondi=0.1, logo=\bcinfo]{Keywords}
\begin{itemize}
\item \ttfamily{CORRECTION ON CONVECTION VELOCITY} (\ttfamily{= NO}, default option)
\end{itemize}
\end{bclogo}


\subsection{Initial and boundary conditions for sediment concentrations}

\subsubsection{Initial conditions}

The initial values of volume concentration for each class can be either imposed
in the subroutine \texttt{condim\_susp.f} or specified in the steering file through the keyword
\texttt{INITIAL SUSPENSION CONCENTRATIONS}. It will not be used if \texttt{EQUILIBRIUM INFLOW CONCENTRATION = YES}.

\subsubsection{Boundary conditions}
For boundary conditions, the concentration of each class can be
specified in the steering file through keyword \texttt{CONCENTRATION PER CLASS AT BOUNDARIES}. 
To avoid unwanted erosion or sedimentation at the entrance of the domain, it may be also convenient to use keyword \texttt{EQUILIBRIUM INFLOW CONCENTRATION = YES} to specify the value of the concentration at inflow, according to the choice of the \texttt{REFERENCE CONCENTRATION FORMULA}. The depth-averaged equilibrium concentration is then calculated assuming equilibrium concentration at the bed and a Rouse profile correction for the $F$ factor. 
Input concentrations can be also directly specified in the subroutine \texttt{conlit.f}.


