%==================================
%==================================
\section{Introduction}
%==================================
%==================================
%==================================
\subsection{A word of caution}
%==================================
This document contains information about the quality of a complex modelling tool. Its purpose is to assist the user in assessing the reliability and accuracy of computational results, and to provide guidelines with respect to the applicability and judicious employment of this tool. This document does not, however, provide mathematical proof of the correctness of results for a specific application. The reader is referred to the License Agreement for pertinent legal terms and conditions associated with the use of the software.

The contents of this validation document attest to the fact that computational modelling of complex physical systems requires great care and inherently involves a number of uncertain factors. In order to obtain useful and accurate results for a particular application, the use of high-quality modelling tools is necessary but not sufficient. Ultimately, the quality of the computational results that can be achieved will depend upon the adequacy of available data as well as a suitable choice of model and modelling parameters.
% 
\subsection{Document plan}
%==================================
This report constitutes the \rel validation note of \tomawac, a module in the TELEMAC system
for simulating water and solute movement in three-dimensional variably-saturated porous media.

\estel software deals mainly with two phenomena:
\begin{list}{-}{}
  \item [-] \textbf{Flow} : the water flow in the soils, saturated or not ;
  \item [-] \textbf{Transport} : the migration of species in the liquid phase.
\end{list}

\estel is often related to the safety assessment studies of nuclear waste disposals.
In this context, the water flow in the soils, together with a large number of other phenomena, affects the migration of the radioactive elements from the container up to the domain boundaries.
In order to give a correct evaluation of the safety assessment of the waste disposal, the code should take into account and correctly reproduce all these physical processes.

Therefore, the \estel code has been verified through a large number of test cases,
including analytical solutions, benchmarking activities and physical validation.

In the first part of the document, an overview of \estel is given to the reader
when \estel special features are documented in the second part of this document.
Chapter 3 specifies the test cases selected for the \estel validation purpose.
Then, the numerical simulations, carried out for this validation activity, will be largely illustrated in chapter 4.
For every considered test case a detailed description will be given,
focusing in particular on the geometry of the domain, both physical and numerical parameters
and an analysis of the results, including comparisons with reference solutions.
A general conclusion closes the document.

%==================================
\subsection{Validation layout}
% %==================================

This validation is presented hereafter using a \textit{validation sheet form},
each sheet detailing the physical concepts involved, the physical and numerical parameters used and comparing both numerical and reference solutions.
Then, each sheet displays the following informations:
\begin{list}{-}{}
\item [-] \textbf{Purpose \& Problem description} : These first two parts give reader short details about the test case, the physical phenomena involved and specify how the numerical solution will be validated;
\item [-] \textbf{Reference} : This part gives the reference solution we are comparing to and explicits the analytical solution when available;
\item [-] \textbf{Physical parameters} : This part specifies the geometry,
details all the physical parameters used to describe both porous media (soil model in particularly) and
solute characteristics (dispersion/diffusion coefficients, soil $\equiv$ pollutant interactions...);
\item [-] \textbf{Geometry and Mesh} : This part describes the mesh used in the \estel computation;
\item [-] \textbf{Initial and boundary conditions} : this part details both initial and boundary conditions used to simulate the case ;
\item [-] \textbf{Numerical parameters} : this part is used to specify the numerical parameters used
(adaptive time step, mass-lumping when necessary...);
\item [-] \textbf{Results} : we comment in this part the numerical results against the reference ones,
giving understanding keys and making assumptions when necessary.
\end{list}
%
\bigskip
%
For more information on the physical/mathematical/numerical concepts used in \estel,
the reader may refer to the \estel \rel principle note \cite{principlenote}.
Concerning \estel activities in the context of the long-term safety assessment of a nuclear waste disposal,
reader may refer to the \estel \rel qualification note \cite{qualificationnote}

\clearpage
%==================================
%==================================
\section{Presentation}
%==================================
%==================================
\subsection{General}
%==================================
\estel is the \textit{groundwater} module in the TELEMAC system,
an hydroinformatic platform developed by the \textit{National Laboratory of Hydraulic and Environment}, EDF R\&D(\cite{JMH1} \&  \cite{JMH2}).

\vspace*{5pt}

For solving the groundwater flow and transport problems,
\estel is based on a set of different libraries within the TELEMAC system:
% 
\begin{list}{-}{}
\item[-] \textbf{ESTEL-3D} : kernel module solving the \textit{Richards} and \textit{Transport} equations in variably saturated porous media;
\item[-] \textbf{BIEF} : the finite element library allows \estel to create vectors and matrices according to the mesh definition.
This library deals also with the usual arithmetic operations such as dot-product, matrix-vector product and a large number of solvers are available for the users;
\item[-] \textbf{DAMOCLES} : this library provides command language tools for the \estel input files;
\item[-] \textbf{SPECIAL} : this library is in charge of tracing system calls/signals;
\item[-] \textbf{PARALLEL} : this library allows the use of the \textit{Message-Passing Interface} in parallel computing.
A mesh partitioner, based on the \textit{METIS} library is also available;
\item[-] \textbf{PARAVOID} : void library when the \textit{MPI} library is not available;
\item[-] \textbf{TELEMAC2D \& SISYPHE} : physical modules in the TELEMAC system who may be called when modelling surface/subsurface exchanges.
This special feature is still on development and is not released in the \rel version of \estel.
\end{list}

%==================================
\subsection{Capabilities}
%==================================
 In order to give an accurate view of the code,
the main options and capabilities of \estel version \rel are illustrated in the following pages.
% 
We remind that \estel \rel is a modelling tool for simulating water movement as well as pollutant transport in three-dimensional variably-saturated porous media.
Some of \estel important features are then:
%
\begin{list}{-}{}
\item[$\bullet$] \textbf{Generalities} :
%
  \begin {itemize}
% 
    \item [-] \estel \rel solves the Richards equation in three-dimensional variably saturated porous media, in transient or steady-state mode. It also solves the transport equation, taking into account the convection, the diffusion/dispersion, the decay, and, when necessary, a simplified geochemistry (Kd approach and precipitation) \cite{principlenote}. Different numerical schemes for solving the flow problem are available, and various retention models are implemented (van Genuchten, Brooks \& Corey, Haverkamp). The transport problem could be solved using two different methods: the SUPG technique and a lagrangian method (no longer supported). We should mention that surface/sub-surface coupling is not available in the \rel release;

    \item [-] The code is built up by a core (the solver) and some functions managing the I/O operations (mesh reading and results writing). The code does neither have an integrated mesh generator nor a post-processing interface, but it is compatible with various mesh formats and post-treatment tools;

    \item [-] The code results could be exploited with the Tecplot software (binary file), or in the SALOME platform (MED file), or through the analysis of the ASCII files;

    \item [-] The initial and boundary conditions, together with some operations on the mesh, on the soils or solute definitions and special outputs, are managed through various user subroutines (FORTRAN). There is currently no user-interface for managing these operations.
% 
   \end{itemize}
\vspace{5pt}
% 
\item[$\bullet$] \textbf{Mesh} :
%
  \begin {itemize}
    \item [-] The code manages only tetrahedral conforming meshes. Two mesh formats are readable by the code: the \textit{universal} format (\textit{.unv}; e.g. generated by the \textit{Ideas} solver in the ANSYS ICEMCFD software), and the \textit{MED} format (\textit{.med}; e.g. the meshes generated by the SALOME platform). Additional informations to the complementary file have to be added (see \cite{userguide})
  \end{itemize}
%
\vspace{5pt}
%
\item[$\bullet$] \textbf{ Capillarity and Conductivity models} :
% 
Various retention models are implemented in the code:
  \begin {itemize}
    \item [-] {\bf Saturated} : this model can be used only when the porous media is assumed to be fully saturated during the whole simulation ;
    \item [-] {\bf Van Genuchten model with Mualem condition} : the most widely used retention model. This model is consistant with the \textit{Unsoda} soil database. However, we should mention that this model may be not defined due to some mathematical inconsistencies ;
    \item [-] {\bf Brooks \& Corey model} : also widely used ;
    \item [-] {\bf Haverkamp model} : seldom used ;
    \item [-] {\bf Tracy model} : theoretical model, only used for validation and model testing.
  \end {itemize}
%
\vspace{5pt}
% 
%   \item [$\bullet$] {\bf Potentialit\'es physiques de base}
%   \begin {itemize}
%     \item [-] propri\'et\'es physiques variables
%     \item [-] possibilit\'e d'imposer une loi utilisateur pour la viscosit\'e
%               turbulente
%     \item [-] mod\'elisation de la variance associ\'ee \'e un scalaire
%     \item [-] pertes de charge
%     \item [-] termes sources pour les \'equations de la masse, de la quantit\'e
%               de mouvement des grandeurs turbulentes 
%               et des scalaires suppl\'ementaires (tels que la 
%               temp\'erature, l'enthalpie, des traceurs passifs, etc.)
%     \item [-] calcul et post-traitement automatique de la pression totale ({\em
%               i.e.} prenant en compte la pression hydrostatique) 
%   \end {itemize}
%
\vspace{5pt}
%
\item[$\bullet$] \textbf{RWPT module} :
A lagrangian module for computing species migration
in variably saturated porous media has been implemented in \estel but is no longer maintained.
This module uses the \textit{Random Walk Particle Tracking} method for solving the transport equation.
%         \begin{itemize}
%           \item[$\rightsquigarrow$] prise en compte de la dispersion stochastique
%           \item[$\rightsquigarrow$] adaptation du pas de temps
%           \item[$\rightsquigarrow$] interaction particule/paroi
%        \end{itemize}

% On pourra se reporter \`a la note \cite{ref_intro_LAGR}.
% 
\vspace{5pt}
%
\item[$\bullet$] \textbf{Parallelism} :
%
An \estel simulation could be executed in a shared or distributed memory parallel machine (e.g cluster).
Almost all of the \estel features are compatible with this important feature.
We may only except special features using the \textit{MED} format. \\
The mesh is partitioned as a pre-processing step with the METIS software,
while the communications between subdomains are managed by the \textit{MPI} technique.\\
At the present time, the partitioning pre-processor deals only with meshes generated in the \textit{universal} (or \textit{IDEAS}) format (.unv).
Current research are still ongoing to achieve the best performance on multicores or heterogeneous machines.
The \estel \rel only works with the \textit{MPI} library but future releases would support multithreading tasks (e.g. Open-MP).
%
\vspace{5pt}
%
\item[$\bullet$] \textbf{Time discretization and time step} :
% 
Various time step treatments are available in the code,
according to the physics of the problem (flow, transport, variably saturated or not...)
\begin{itemize}
\item[-] {\bf Flow time step} - \textit{constant/adaptive/user-defined time step} : When soils are fully saturated, the constant time step in flow is sufficient to compute rapidly accurate solutions.
As retention models are often highly non-linear, this option is necessary when modelling water movement in unsaturated soils;
\item[-] {\bf Transport time step} - \textit{constant/adaptive/user-defined} : Although the problem is not as difficult to solve as the flow problem,
different time step treatments may be used for some CPU time (user-defined time step) or following some physical quantities (Courant number).
\item[-] {\bf 1D \& 3D Outputs time step} - \textit{constant/user-defined} : 3D files often become heavily to manipulate and spend a lot of time to be written.
Simplified outputs (commonly called 1D outputs) have been introduced since \estel V6P1. 1D outputs include:
\begin{list}
 - time-dependent global quantities such as water and tracer masses, mass-balance errors
 - both hydraulic and solute fluxes (convective, dispersive, diffusive and total)
 - user-defined probes of main quantities (pressure head, velocities, moisture content, concentration in both liquid and solid phase)
\end{list}
\end {itemize}
\vspace{5pt}
% 
\item[$\bullet$] \textbf{Boundary conditions} :
  \begin {itemize}
    \item [-] Dirichlet condition - imposed value of the variable;
    \item [-] Neumann condition - imposed flux of the variable through the boundary;
    \item [-] Cauchy condition - mixed information both on value and flux;
    \item [-] Free exit - only for the transport equation, assuming a purely convective flux.
  \end{itemize}
 \vspace{5pt}
%
\item[$\bullet$] \textbf{Sources and sinks} :
%
  \begin {itemize}
    \item [-] Sources and sinks are implemented for both groundwater flow and transport problems.
  \end{itemize}
 \vspace{5pt}
%
\item[$\bullet$] \textbf{Linear system solvers} :
%
  \begin {itemize}
    \item [-] For the water flow problem, a fixed-point iteration (Picard) has been developed for solving this non-linear problem. For solving the linear system, the Conjugated Gradient solver with a CROUT preconditioning is often used. As default values, the required precision is $10^{-5}$ (absolute error) and the maximal number of iterations is set equal to $500$.
    \item [-] For the transport equation, according to the non-symmetry of the matrix, the Conjugated Gradient on the normal equation solver with a CROUT or diagonal (in parallel simulation) preconditioning is used. The required precision is $10^{-5}$ and the maximal number of iterations is set equal to $500$.
  \end{itemize}
% \end {itemize} 
% 
\item[$\bullet$] \textbf{Post-processing features} :
%
  \begin {itemize}
    \item [-] \estel code includes post-processing features such as time histories of global quantities (volume of water, mass added due to sources...) or of user-defined points. Fluxes computations are also available when defining border/inner user-surfaces. All of these features are compatible with the parallel mode.
  \end{itemize}
%
% {\bf Remarque : }
% La valeur par d\'efaut de la {\bf pr\'ecision asbolue} pour la r\'esolution du sch\'ema it\'eratif de Picard est volontairement relativement forte. En effet, les temps de calculs sont g\'en\'eralement relativement longs dans ce cas alors que peu de diff\'erences significatives sur les r\'esultats sont g\'er\'enralement constat\'ees.
%
\vspace{5pt}
%
\end {list} 
% 
\clearpage
% %==================================
\section{Validation}
% %==================================
% As \estel is mainly used 
%
%==================================
\subsection{Evolution compared to the previous release}
%==================================
The main developments from 5.9 to \rel are detailed hereafter:
\begin{itemize}
  \item [$\bullet$] {\bf New physical features~:} 
    \begin{itemize}
      \item [-] the solute transport module has been released (SUPG formulation);
      \item [-] a simplified geochemistry is available of specifying {soils$\equiv$pollutant} interactions : linear isotherm (Kd hypothesis), first-order decay, precipitation and dissolution;
      \item [-] filiation;
      \item [-] mass-balance in the transport module.
    \end{itemize}
  \item [$\bullet$] {\bf HPC features~:} 
    \begin{itemize}
      \item [-] parallel optimization in the \textit{BIEF} library. Extra-communication have been added for ensuring parallel robustness when the initial conditions are specified by soil layers;
      \item [-] GMRES is currently available (useful when solving the transport equation but still conflicting with the CROUT preconditioning);
    \end{itemize}
  \item [$\bullet$] {\bf User-friendly features~:} 
    \begin{itemize}
      \item [-] initial conditions can be specified using the soil layer index;
      \item [-] Optimization of the mesh partitioner (C. Denis, Sinetics, EDF R\&D);
      \item [-] \textit{MED} library is support in the TELEMAC system with the PGI, INTEL and GNU fortran compiler. Restarting a simulation with a MED results file is also available;
      \item [-] \estel \rel is fully SALOME compliant (pre/posts).
    \end{itemize}
  \item [$\bullet$] {\bf Documentation~:} 
    \begin{itemize}
      \item [-] documentation management using SVN.
    \end{itemize}
\end{itemize}

%==================================
\subsection{Difference with the previous validation}
%==================================
As the \estel is relatively recent compared to the other modules in the TELEMAC system and
as the contaminant transport module has just been released,
the whole validation document has been reviewed to be representative of the current release of the \estel code.
Then, almost all of the test cases presented hereafter are new and have been chosen in order to:
\begin{itemize}
\item [-] keep a significant number of test cases ;
\item [-] represent at least different groundwater configurations ;
\item [-] represent every physical feature at least one time ;
\item [-] involve an \estel special feature (post-processing, parallelism, adaptive time step...) ;
\item [-] be representative of test cases that can be found in the scientific community
\end{itemize}

% \vspace*{0.5cm}
% 
% Pour permettre d'ajouter des nouveaux cas (nouvelles physiques particuli\`eres notamment) sans
% alourdir la validation, certains cas ont \'et\'e abandonn\'es :

%==================================
\subsection{Potentialities tested}
%==================================
\label{prg_potentialites_testees}%
We should mention that, in particular, the following potentialities have been tested (non exhaustive list).
%
\begin{itemize}
\item [$\bullet$]{\bf Domain, mesh and I/O} 
\begin{itemize}
  \item Fake 1D/2D configurations ;
  \item 3D configuration with/without symmetric plans ;
  \item Use of MED input/output files (tetrahedral mesh);
  \item Use of IDEAS input and Tecplot output files ;
  \item Use of post-processing features (histories of physical quantities, fluxes on user-defined surfaces...) ;
  \item Computation restart (Sequential and MED files only).
\end{itemize}
%
\item [$\bullet$]{\bf Physical potentialities} 
\begin{itemize}
  \item Variably saturated and heterogeneous porous media ;
  \item Multi-components transport with simplified geochemistry ($K_d$, $1^{st}$ order decay, filiation/precipitation, decay chain).
\end{itemize}
%
\item [$\bullet$]{\bf Special features} 
\begin{itemize}
  \item Use of adaptive time step ;
  \item Use of mass-lumping in both groundwater and transport modules ;
  \item Almost all features are compatible with parallelism (domain decomposition).
\end{itemize}
\end{itemize}

% %==================================
% \subsection{Dysfonctionnements not\'es lors de la validation}
% %==================================

% %==================================
% \subsubsection{Dysfonctionnements du Noyau}
% %==================================
% \label{prg_dysfnoyau130}%
% On donne ici la liste des dysfonctionnements concernant le noyau de \estel
% \rel. Pour chacun, on \'evalue le risque qu'il entraîne et on pr\'ecise les cas
% potentiellement impact\'es, les correctifs propos\'es et les tests de validation
% \rel associ\'es.

% \begin{list}{\texttt{$\bullet$}}{}

% \vspace{0.2cm}\item 
% {\bf PROBL\`EME : }Manque les arguments NCEL, NCELET, NFAC, NFABOR et NNOD dans
% \texttt{ustbus.F}\\ 
% {\bf IMPACT : }Impossible de d\'efinir des tableaux utilisateurs en dehors de l'Interface\\
% {\bf CAS IMPACT\'ES : }Tous les cas pass\'es sans interface n\'ecessitant des
% tableaux utilisateurs\\
% {\bf CORRECTIF : }Passage des arguments concern\'es \'e la routine \texttt{ustbus.F}\\
% {\bf CAS DE VALIDATION \rel.1 : }Utilisation de ustbus.F pour HISHIDA et CANAL\_PLAN
% 
% \vspace{0.2cm}\item 
% {\bf PROBL\`EME : }En cas de d\'ecoupage des faces gauches, certaines variables
% (dont NDIMFB=nombre de faces de bord) ne sont pas remises \'e jour\\
% {\bf IMPACT : }Erreur de dimensionnement de certains tableaux et \'ecrasements potentiels\\
% {\bf CAS IMPACT\'ES : }Tous les cas o\'e le d\'ecoupage des faces gauches est activ\'e\\
% {\bf CORRECTIF : }Mise \'e jour des dimensions concern\'ees\\
% {\bf CAS DE VALIDATION \rel.1 : }COUDE180 sur maillage vrill\'e avec red\'ecoupage
% des faces gauches
% 
% \end{list}
%
% %==================================
% \subsubsection{Dysfonctionnements \rel}
% %==================================
% The validation/correction process has been done simultaneously.
% Then, we don't provide comments on

% %==================================
% \subsubsection{Autres modifications apport\'ees au Noyau \rel.1}
% %==================================
% %
% Mis \`a part les corrections des dysfonctionnements not\'es pr\'ec\'edemment, d'autres
% modifications ont \'et\'e apport\'ees au Noyau \rel.1, corrections de probl\`emes ou
% am\'eliorations, d\'etect\'ees lors de calculs hors validation ou lors de
% d\'eveloppements de versions futures. On liste ici les modifications
% principales.

% \begin{list}{-}{}
% % \item Adaptation des routines de lecture du fichier Xml de l'Interface aux
% % modifications apport\'ees par l'Interface \rel.1 sur les donn\'ees du charbon
% % pulv\'eris\'e : le cas CERCHAR a permis de valider le correctif
% % \item Erreur (sans impact) dans l'initialisation des variables DEPALE et CSMAGO
% % \item Modification du test qui arr\'ete le calcul quand le gradient conjugu\'e
% % diverge
% % \item Modification de \texttt{strpre.F} pour la prise en compte d'un d\'eplacement
% % initial d'une structure en interaction fluide/structure par couplage interne
% % (pas de cas test associ\'e, mais correctif test\'e en dehors de la pr\'esente
% % validation)
% % \item Correction d'un probl\'eme de d\'ecalage de face de hauteur de r\'ef\'erence pour
% % la pression totale quand on a des sorties libres et que la pression est impos\'ee
% % en dur par l'utilisateur sur certaines faces de sortie (pas toutes). Cette
% % configuration ne correspond {\em a priori} \'e aucun calcul r\'eel, dans la
% % validation ou en dehors.
% % \item Corrections multiples de commentaires ou nettoyage de variables non utilis\'ees
% \end{list}


%==================================
% \subsection{Dysfonctionnements not\'es sur la version \rel.1}
% %==================================
% Lors de la phase de contre-validation, trois dysfonctionnement ont \'et\'e d\'etect\'es
% sur la version \rel.1 (Noyau, Pr\'eprocesseur ou Interface) :
% \begin{itemize}
% % \item Probl\'eme sur le calcul {\em via} l'Interface des moyennes temporelles de variables
% % situ\'ees dans le tableau PROPCE. Le cas impact\'e (CERCHAR) a
% % \'et\'e pass\'e avec une version corrig\'ee de la routine \texttt{cs\_gui.c}.
% \end{itemize}


% Outre ces \'el\'ements, d'autres dysfonctionnements ont \'et\'e d\'etect\'es en dehors de la
% validation, d'impact mod\'er\'e ou faible :

% \minititre{Impact mod\'er\'e}
% \begin{itemize}
% % \item Correction de l'initialisation de la distance \'e la paroi lorsque l'on
% %   utilise la m\'ethode de calcul direct (d\'econseill\'ee) en mod\`ele de turbulence
% % $k-\omega$ SST 
% \end{itemize}

% \minititre{Impact faible ou nul}
% \begin{itemize}
% % \item Correction d'une erreur de lecture du tableau ETTP lors d'une suite de
% % calculs en lagrangien ; la vitesse vue par les particules est alors mal relue,
% % ce qui entra\'ene uniquement une diminution de l'ordre en temps pour l'it\'eration
% % de reprise de calcul.
% \end{itemize}



% %==================================
% \subsection{Technical points}
% %==================================
% 
% We detail hereafter some useful technical points for understanding this validation document.
% 
% \textbf{Adaptative Time Step} When dealing with unsaturated zone, \estel convergence is facilitated by the use of an adaptative time step. Time step controls 
% 
% \vspace*{5pt}
%  
% \textbf{Mesh quality} We give no information on the mesh quality in our validation sheet, this information may however be known from the mesh tool (ICEM, Salom\'e)
% 
% \vspace*{5pt}
%  
% \textbf{Les nombres adimensionnels} tels que le nombre de Courant (CFL) et le nombre de P\'eclet sont calcul\'es par par rapport au tenseur de dispersion et aux vitesses calcul\'ees par le module hydraulique ou fix\'ees par l'utilisateur. On pourra retenir  que le nombre de Courant caract\'erise le rapport $\frac{U\Delta t}{\Delta x}$ et que le nombre de P\'eclet caract\'erise le rapport des forces de convection sur les forces de dispersion $\frac{U\_\Delta x}{D}$  (o\`u $\Delta t$ repr\'esente  
% le pas de temps, $\Delta x$ la taille des mailles, $U$ la vitesse du fluide et $D$ la dispersion du fluide dans le domaine).
% 
% \vspace*{5pt}
% 
% \textbf{La taille des mailles de bords} sur lesquelles sont impos\'ees une sortie libre peut \^etre augment\'ee afin de favoriser les effets convectifs et ainsi renforcer l'hypoth\`ese que le flux convectif est le seul contributeur \`a la sortie du solut\'e.
% \clearpage
%==================================

% \clearpage
%==================================
% %==================================
% \section{Version corrig\'ee \rel.1 de \estel}
% %==================================
% \label{prg_version131}%
% Lors de la premi\'ere phase de validation de la version \rel.0, plusieurs
% dysfonctionnements d'importance variable ont \'et\'e rencontr\'es. Chaque fois qu'un
% dysfonctionnement bloquant a \'et\'e d\'etect\'e sur un calcul, il a 
% donn\'e lieu \'e un correctif (sous-programme non utilisateur corrig\'e \'e placer
% dans le r\'epertoire FORT du calcul), qui a ensuite \'et\'e utilis\'e d\'es que
% n\'ecessaire dans les calculs suivants\footnote{la biblioth\'eque \rel.0 n'a par
% contre pas \'et\'e modifi\'ee, pour une meilleure tracabilit\'e des corrections}.
% L'impact du dysfonctionnement a aussi \'et\'e
% analys\'e afin de conna\'etre son influence \'eventuelle sur les calculs d\'ej\'e pass\'es
% et d\'efinir la liste des cas \'e passer en contre-validation avec la version \rel.1.
% 
% La liste des dysfonctionnements rencontr\'es est donn\'ee ci-dessous. Certains
% d'entre eux avaient un impact potentiel fort et surtout une probabilit\'e
% d'occurence difficile \'e \'evaluer (notamment un probl\'eme de d\'ecalage d'indice dans
% la num\'erotation du voisinage \'etendu utilis\'e pour le calcul des gradients). Il a
% donc \'et\'e d\'ecid\'e que la phase de contre-validation de la version \rel.1 serait
% plus importante que pr\'evue et que tous les cas tests seraient repass\'es dans au
% moins une configuration.



% \section{Test case selection}
% %==================================
% 
% La phase de validation vise \`a \'evaluer le caract\'ere op\'erationnel de la
% version \rel ainsi que la qualit\'e des 
% sch\'emas num\'eriques et des mod\`eles mis en \oe uvre. 
% Afin d'\^etre capable de distinguer imm\'ediatement les dysfonctionnements
% des d\'efauts inh\'erents aux mod\`eles, il est n\'ecessaire de choisir
% des cas bien document\'es et sur lesquels on dispose d'une solide 
% exp\'erience. En outre, il a \'et\'e jug\'e utile d'inclure \`a la validation
% des cas plus industriels pour lesquels l'objectif est moins ``analytique'' et
% plus tourn\'e vers l'\'evaluation du comportement du code en configuration
% d'\'etude plus r\'ealiste. A noter qu'aucun plan de validation n'a \'et\'e effectu\'ee au pr\'ealable.

%==================================
\subsection{Architectures}
%==================================
\label{prg_archi}%
Even if \estel has been designed to be run on both Windows and Unix-like architectures,
the code is more often used on Linux. We also remind TELEMAC users that the TELEMAC system assumes the input and output files to be written as \textit{Big\_Endian}.
Current architectures as well as compiler options used for development or validation purposes are detailed hereafter (tables \ref{tab_arch62} and \ref{tab_compilo62}).
\begin{table}[!h]
\begin{center}
\renewcommand{\arraystretch}{1.2}
\small
\begin {tabular} {c|c}
    \hline 
       Architecture &  Compiler \\
    \hline
    \hline 
PC Linux Calibre 7 & ifort, pgf90, gfortran, nag \\
PC Linux Fedora 13/14 & gfortran \\
PC Windows 7 & ifort, gfortran, g95 \\
Cluster Linux Ivanoe & ifort \\
Cluster BlueGeneQ & xlf90 \\
\end {tabular}
\normalsize
\caption{Architecture currently supported by \estel \rel}\label{tab_arch62}
\end{center}
\end{table}
% 
\begin{table}[!h]
\begin{center}
\renewcommand{\arraystretch}{1.2}
\tiny
\begin {tabular} {l|l|l|l|l|l}
    \hline    
%-----------------------------------------------------------------------------------
    Compiler      & Ifort          & Nag & Xlf90 & Gfortran       & G95 \\
                  & (Linux/Windows)&  &  & (Linux/Windows) & (Windows) \\
    \hline
    \hline  
    Version       & V10.1 & V5 & V14 & V4.4.5 & ?? \\
 FC\_OPT\_COMPIL  & -c -O3 & & -c -O3 -qtune=qp -q64 ... & -c -O3  \\
 Endianess        & -convert big\_endian & $\times$ & & -fconvert=big-endian \\
 Preprocessing C  & -cpp -D<>& -fpp -D<> & -qsuffix=cpp=f90 -WF,-D<> & -fpp & -fpp \\
 LK\_OPT\_NORMAL  & -lstdc{\tiny ++} & & &-lstdc{\tiny ++} &\\
 LK\_LIB\_SPECIAL & MED,Tecplot & & & Med,Tecplot & \\
  LIBS\_MPI       & MPICH or OpenMPI &  Mpich & PAMI & MPICH or OpenMPI \\
    \hline  
  \end {tabular}                  
\normalsize
\caption{List of \estel compatible compilers and their options for the \estel \rel validation}\label{tab_compilo62}
\end{center}
\end{table}

\newpage

\subsection{Test case report}
%==================================
\estel has been verified through a large number of test cases, presented in table \ref{tab_valid62}, including analytical solutions, benchmarking activities and physical validation.

% %==================================
% \subsection{Test cases description}
% %==================================
\begin{table}[h!]
\begin{center}
\renewcommand{\arraystretch}{1.2}
\tiny
\begin {tabular} {l|c}
% 
    \hline 
    \multicolumn{2}{c}{\textbf{Comparison with analytical solutions}}\\       
    \hline 
    \hline 
%------------------------------------------------------------------------------------
TRACY$_{1D}$ (no gravity) &
Variably saturated flow in a horizontal soil column (no gravity)\\
TRACY$_{1D}$ (with gravity) &
Variably saturated flow in a horizontal soil column (with gravity)\\
TRACY$_{3D}$        &
Variably saturated flow in a 3D column soil\\
CELIA$_{1D}$ &
Variably saturated flow in a horizontal soil column with Haverkamp model\\
ST2A        &
2D analytical solute transport solution (first-type boundary condition)\\
FLUX$_{1D}$        &
1D analytical solute transport solution (third-type boundary condition)\\
FLUX$_{3D}$        &
3D analytical solute transport solution (third-type boundary condition)\\
SOURCE        &
3D analytical solute transport for instantaneous source\\
CHAIN        &
Transport of a decay chain\\ 
FLUX (Post-Processing) &
Validation of fluxes computation on user-defined surfaces\\
\hline
\multicolumn{2}{c}{\textbf{Surface - SubSurface coupling}}\\
\hline
\hline
COFFEE-FILTER        &
Infiltration in subsurface in coupled mode \\
 & Comparison with analytical solution\\
\hline
\multicolumn{2}{c}{\textbf{Benchmarks}}\\
\hline
\hline
HYDRUS$_{1D}$        &
Transient unsaturated flows and transport in a 1D column of homogeneous soil. \\
 & Comparison with results from HYDRUS$_{1D}$\\
HYDRUS$_{3D}$        &
Saturated flows and transient transport in a 3D column of homogeneous soil. \\
 & Comparison with results from HYDRUS$_{3D}$\\
PORFLOW            &
3D solute transport for continuous diffusive boundary solute source\\
& Comparison with PORFLOW results\\
PRECIPITATION            &
Precipitation and dissolution of migrating elements\\
 & Comparison with PORFLOW results \\
HYDROCOIN          &
Comparison with HYDROCOIN project results\\
SINK-SOURCE        &
Demonstration test case of sink/source terms (flow and transport)\\
\hline
\multicolumn{2}{c}{\textbf{Non linear sorption}}\\
\hline
\hline
BATCH$^{Equilibrium}$           &
Sorption at equilibrium in a batch with different models\\
 & Analytical solution\\
BATCH$^{Kinetic}$           &
Sorption with kinetic in a batch with different models\\
 & Analytical solution \& Benchmark with HYDRUS$_{1D}$\\
COLUMN$^{Equilibrium}_{1D}$           &
Sorption at equilibrium in a column with infiltration (different sorption models)\\
 & Benchmark with HYDRUS$_{1D}$\\
COLUMN$^{Kinetic}_{1D}$           &
Sorption with kinetic in a column with infiltration (different sorption models)\\
 & Benchmark with HYDRUS$_{1D}$\\
\hline
\multicolumn{2}{c}{\textbf{Waste repository studies}}\\
\hline
\hline
\todo{GRS/IRSN}           &
\todo{Radionuclide migration in a simplified nuclear waste repository}\\
 & \todo{Comparison with results from MELODIE [IRSN]}\\
COUPLEX-1 $_{2D}$          &
Radionuclide migration in a simplified nuclear waste repository\\
 & Benchmark from GNR MOMAS\\
URT           &
Radionuclide migration in a complex nuclear waste repository\\
 & Qualification study\\
COUPLEX-2D           &
Radionuclide migration in a complex nuclear waste repository with precipitation\\
 & Qualification study\\

\hline
    \multicolumn{2}{c}{\textbf{Physical validation}}\\
    \hline
\hline
M27               &
Field experiments through the vadose zone (ANR-TRANSAT project) \\
Large Infiltrometer &
Field experiments through the vadose zone (water infiltration, solute transfer, pumping) (ADEME CAPHEINE project) \\
\end {tabular}
\normalsize
\caption{Description of the \estel \rel test cases}\label{tab_valid62}
\end{center}
\end{table}

In the following pages, we present the \textit{validation report} for every test case using a \textit{sheet form}, detailing the physical concepts involved, the physical and numerical parameters used and comparing both numerical and reference solutions.
